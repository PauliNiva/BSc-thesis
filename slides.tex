\documentclass{beamer}

\usepackage[utf8]{inputenc}

\title{Ehrenfeucht--Fraïssé-peleistä}
\author{Pauli Niva}
\date{06.04.2016}



\usepackage{mathtools}
\usepackage {tikz}
\usetikzlibrary {positioning}
%\usepackage {xcolor}
\definecolor {processblue}{cmyk}{0.96,0,0,0}

\begin{document}

\frame{\titlepage}

\begin{frame}
\frametitle{Intro}
Ehrenfeucht--Fraïssé-peli on malliteorian menetelmä, joka perustuu struktuurien samankaltaisuuden tarkasteluun. \pause Malliteorian menetelmistä suurin osa ei kuitenkaan toimi äärellisten mallien tapauksessa. \pause EF-peli on yksi harvoista menetelmistä, joka toimii myös äärellisillä malleilla. \pause Äärellisten mallien teoria on matemaattisen logiikan ja teoreettisen tietojenkäsittelytieteen, tarkemmin laskennan teorian osa-alue.
\end{frame}

\begin{frame}
\frametitle{Sovellusalueita}
EF-peli on teoreettinen työkalu jota pääasiassa käytetään määriteltävyyskysymyksiin ja todistusten apuna.
Äärellisten mallien teoriaa ja sen menetelmää EF-peliä voidaan soveltaa tietojenkäsittelytieteessä muun muassa seuraavilla osa-alueilla:
\end{frame}

\begin{frame}
\frametitle{Verifikaatio}
Äärelliset mallit voidaan koodata
\begin{itemize}
\item<1-> puiksi
\item<2-> verkoiksi
\item<3-> sanoiksi
\end{itemize}
\pause \pause \pause
eli niitä voidaan käyttää laskennan olioina ja täten niillä voidaan kuvata äärellistilallisia systeemejä ja tutkia näiden toiminnan oikeellisuutta.
\end{frame}

\begin{frame}
\frametitle{Tietokantateoria}
Relationaalinen malli samaistaa tietokannan äärellisen relaationaalisen struktuurin kanssa.
\pause
Formaalin kielen kaavat voidaan ajatella ohjelmina, jotta niiden merkitystä struktuurissa voidaan arvioida.
\pause
Ja toisinpäin, voidaan esittää jonkin laskennallisen vaativuusluokan kyselyitä jollakin formaalilla kielellä
\end{frame}

\begin{frame}
\frametitle{Laskennan vaativuus}
Tarjoaa laskennan vaativuusluokkien loogisen karakterisoinnin ja mahdollistaa vaativuusteoreettisten tulosten todistamisen tätä kautta.
\pause
\\
Esimerkiksi $P = NP$ -ongelma redusoituu kysymykseksi: onko kahdella kiintopistelogiikalla sama ilmaisuvoima äärellisissä malleissa 
\end{frame}

%frame1
\begin{frame}
\frametitle{Historiallinen tausta}
\begin{itemize}
\item<1->1930-luvulla Alfred Tarski esittelee elementaarisen ekvivalenssin käsitteen
\item<2->1946 Tarski Princetonissa pidetyssä konferenssissa joka käsittelee ongelmia matematiikassa:

On olemassa algebroja jotka eivät ole isomorfisia, mutta joita ei kuitenkaan voi erottaa toisistaan näiden aritmeettisten ominaisuuksien perusteella; Tarvitaan teoria algebrojen aritmeettiselle ekvivalenssille joka on käsitteenä yhtä perustavanlaatuinen kuin isomorfismi.
\item<3->1954 Roland Fraïssé esittelee osittaisisomorfismin, jonka avulla saadaan elementaarinen ekvivalenssi predikaattilogiikalle
\item<4->1961 Andrzej Ehrenfeucht muokkasi Fraïssén menetelmästä peliteoreettisen version
\end{itemize}
\end{frame}

%frame2
\begin{frame}
\frametitle{Relaatiot}
Joukon $X$ \textit{kaksipaikkainen relaatio} $R$ on mikä tahansa joukko joukon $X$ alkioista muodostettuja pareja $(x, y)$, joiden molemmat alkiot ovat joukossa $X$, eli $R \subset X^2$. \pause Jos $(x, y) \in R$, sanotaan, että $x$ on $y$:n kanssa \textit{relaatiossa} $R$.
\end{frame}

%frame3
\begin{frame}
\frametitle{Mallit}
Olkoon $L$ aakkosto ja olkoon $M$ epätyhjä joukko. Tällöin $L$\textit{-malli} koostuu seuraavista:
\begin{itemize}
\item<1-> Joukosta $M$.
\item<2-> Relaatioista $R^M \subset M$, jokaiselle relaatiosymbolille $R \in L$, $\#R = n$, jossa $n\in \mathbb{Z}_+$.
\item<3-> Vakioista $c^M \in M$, jokaiselle vakiosymbolille $c \in L$.
\item<4-> Funktioista $f^M: M^m \rightarrow M$, jokaiselle funktiosymbolille $f \in L$, $\#f = m$, jossa $m\in \mathbb{Z}_+$.
\end{itemize}
\pause \pause \pause \pause
Relaatiota $R^M$ sanotaan relaatiosymbolin $R$ \textit{tulkinnaksi mallissa} $M$, funktiota $f^M$ sanotaan funktiosymbolin $f$ \textit{tulkinnaksi mallissa} $M$, ja alkiota $c^M$ kutsutaan vakiosymbolin $c$ \textit{tulkinnaksi mallissa} $M$.
\end{frame}

\begin{frame}
\frametitle{Mallit jatkuu}
Malli siis antaa aakkoston $L$ symboleille \textit{semantiikan} eli merkityksen sekä kontekstin, jossa formaalin kielen lauseet voivat olla tosia tai epätosia. \pause Samaa merkintää $M$ käytetään sekä mallista kokonaisuutena että mallin \textit{universumista} eli mallin alkioiden joukosta.
\end{frame}

\begin{frame}
\frametitle{Symbolin ja sen tulkinnan erosta}
Relaatiosymbolin $R$ ja sen tulkinnan $R^M$ välistä eroa voidaan havainnollistaa esimerkiksi sanan ``tietokone'' ja tietokoneen välisellä erolla. \pause Sana ``tietokone'' on suomen kieltä, joka koostuu yhdeksästä merkistä ja sitä voidaan käyttää muodostettaessa suomenkielisiä lauseita. \pause Tietokone taas on fyysinen laite, joka käsittelee tietoa ohjelmointinsa mukaisesti, eikä sitä voida käyttää suomen kielen lauseiden osana. \pause Lause ``Tietokoneeni on Mac'' on totta tai epätotta, riippuen siitä, mihin nimenomaiseen tietokoneeseen sana ``tietokone'' viittaa.
\end{frame}

%frame4
\begin{frame}
\frametitle{Isomorfiset mallit}
Kaksi mallia $A$ ja $B$ yli saman äärellisen aakkoston $L$ ovat isomorfisia $(A \cong B)$ jos on olemassa isomorfismi $A$:lta $B$:hen, eli on olemassa bijektio $f: A \rightarrow B$, joka säilyttää relaatiot ja vakiot.
\end{frame}

\begin{frame}
\frametitle{Elementaarinen ekvivalenssi}
Siinä missä isomorfismi kuvailee kahden mallin rakenteellista samanlaisuutta, niin elementaarinen ekvivalenssi puolestaan vertailee malleja suhteessa käytettyyn kieleen. \pause Olkoon $L$ aakkosto joka muodostaa kielen $K$. Olkoon $A$ ja $B$ kummatkin $L$-malleja. $A$:ta ja $B$:tä sanotaan \textit{elementaarisesti ekvivalenteiksi}, jos kaikilla lauseilla $S \in K$ pätee $A \vDash S \Longleftrightarrow B \vDash S$. Tätä merkitään $A \equiv B$
\end{frame}

\begin{frame}
\frametitle{Elementaarinen ekvivalenssi jatkuu}
Jos $L$-mallit $A$ ja $B$ ovat isomorfiset, niin ne ovat elementaarisesti ekvivalentit. \pause

On huomattava, että tämä ei päde toisinpäin. Mallien $A$ ja $B$ välinen elementaarinen ekvivalenssi ei kerro mitään mallien isomorfisuudesta. 
\end{frame}

\begin{frame}
EF-pelissä ideana on, että peli on kahdelle pelaajalle, joita kutsutaan nimillä Pelaaja I ja Pelaaja II. Peliä pelataan kahdella mallilla $A$ ja $B$, joilla on sama aakkosto. Pelaaja II haluaa osoittaa, että kyseiset mallit ovat jossain määrin samankaltaiset, kun taas Pelaaja I haluaa osoittaa, että mallit ovat erilaiset. Pelissä on äärellinen määrä vuoroja ja vuorojen määrä on alussa sovittu.
\end{frame}

\begin{frame}
\frametitle{EF-pelin kierroksen kulku}
Merkitään pelattavien kierrosten määrää luvulla $k \in \mathbb{Z}_+$. EF-peliä pituudeltaan $k$-kierrosta malleilla $A$ ja $B$ merkitään $EF_k(A, B)$. Pelin $EF_k(A, B)$ mielivaltaisen kierroksen $i \in \{1, \ldots, k\}$ kulku on seuraavanlainen: 
\begin{itemize}
\item<2-> Ensin Pelaaja I valitsee toisen malleista $A$ tai $B$ sekä jonkin alkion $a_i \in A$ tai $b_i \in B$ tästä mallista.
\item<3-> Tämän jälkeen Pelaaja II valitsee malleista sen, jota Pelaaja I ei valinnut ja valitsee tästä mallista jonkin alkion.
\end{itemize}
\end{frame}

\begin{frame}
\frametitle{EF-pelin voittaminen}
Olkoon $a = (a_1, \ldots, a_i)$ mallista $A$ valitut alkiot ja $b = (b_1, \ldots, b_i)$ mallista $B$ valitut alkiot mielivaltaisella kierroksella $i$. \pause Pelaajan II voittaa jos ja vain jos jokaisella kierroksella $i \leq n$ pari $(a, b)$ määrää osittaisen isomorfismin $A \rightarrow B$ eli on olemassa kuvaus $h: A \rightarrow B$, siten että $a \in A \mapsto b \in B$ \pause Muussa tapauksessa Pelaaja I voittaa.
\end{frame}

\begin{frame}
\frametitle{Voittostrategia}
Strategia on joukko sääntöjä, joiden mukaan pelaaja tekee valintansa toisen pelaajan valinnasta riippuen. \pause EF-pelissä kummallakin pelaajalla on koko ajan tiedossa mallit, niiden rakenne ja jo tehdyt valinnat, eli peli on \textit{täydellisen informaation peli}. \pause Strategiaa, jota seuraamalla pelaaja voittaa pelin riippumatta mitä valintoja toinen pelajaa tekee kutsutaan \textit{voittavaksi strategiaksi}. \pause Jos Pelaaja II:lla on voittava strategia EF-pelissä $EF_k(A, B)$, niin tätä merkitään $A \sim_k B$.
\end{frame}

\begin{frame}
\begin {center}
\begin {tikzpicture}[-latex ,auto ,node distance =4 cm and 5cm ,on grid ,
semithick ,
state/.style ={ circle ,top color =white , bottom color = processblue!20 ,
draw,processblue , text=blue , minimum width =1 cm}]
\node[state] (x_1) {$x_1$};
\node[state] (x_2) [above =of x_1] {$x_2$};
\node[state] (y_2) [above right =of x_1] {$y_2$};
\node[state] (y_1) [right =of x_1] {$y_1$};
\node[state] (y_3) [right =of y_1] {$y_3$};
\path (x_1) edge (x_2);
\path (x_2) edge (x_1);
\path (y_1) edge (y_2);
\path (y_2) edge (y_1);
\path (y_3) edge (y_1);
\path (y_1) edge (y_3);
\path (y_2) edge (y_3);
\path (y_3) edge (y_2);
\node[align=left, below] at (0.1,-.9)%
{Verkko \textit{X}};
\node[align=right, below] at (7.5,-.9)%
{Verkko \textit{Y}};
\end{tikzpicture}
\end{center}
Kierros 1:
\begin{itemize}
  \item<2> Pelaaja I voi valita solmun kummasta verkosta tahansa.
\end{itemize}
\end{frame}

\begin{frame}
\begin {center}
\begin {tikzpicture}[-latex ,auto ,node distance =4 cm and 5cm ,on grid ,
semithick ,
state/.style ={ circle ,top color =white , bottom color = processblue!20 ,
draw,processblue , text=blue , minimum width =1 cm}]
\node[state] (x_1) {$x_1$};
\node[state] (x_2) [above =of x_1] {$x_2$};
\node[state] (y_2) [above right =of x_1] {$y_2$};
\node[state] (y_1) [right =of x_1] {$y_1$};
\node[state] (y_3) [right =of y_1] {$y_3$};
\path (x_1) edge (x_2);
\path (x_2) edge (x_1);
\path (y_1) edge (y_2);
\path (y_2) edge (y_1);
\path (y_3) edge (y_1);
\path (y_1) edge (y_3);
\path (y_2) edge (y_3);
\path (y_3) edge (y_2);
\node[align=left, below] at (0.1,-.9)%
{Verkko \textit{X}};
\node[align=right, below] at (7.5,-.9)%
{Verkko \textit{Y}};
\end{tikzpicture}
\end{center}
Kierros 1:
\begin{itemize}
  \item Jos Pelaaja I valitsee solmun verkosta $X$, Pelaaja II valitsee vastinpariksi verkosta $Y$ solmun $y_1$, muulloin Pelaaja II valitsee vastinpariksi solmun $x_1$.
\end{itemize}
\end{frame}

\begin{frame}
\begin {center}
\begin {tikzpicture}[-latex ,auto ,node distance =4 cm and 5cm ,on grid ,
semithick ,
state/.style ={ circle ,top color =white , bottom color = processblue!20 ,
draw,processblue , text=blue , minimum width =1 cm}]
\node[state] (x_1) {$x_1$};
\node[state] (x_2) [above =of x_1] {$x_2$};
\node[state] (y_2) [above right =of x_1] {$y_2$};
\node[state] (y_1) [right =of x_1] {$y_1$};
\node[state] (y_3) [right =of y_1] {$y_3$};
\path (x_1) edge (x_2);
\path (x_2) edge (x_1);
\path (y_1) edge (y_2);
\path (y_2) edge (y_1);
\path (y_3) edge (y_1);
\path (y_1) edge (y_3);
\path (y_2) edge (y_3);
\path (y_3) edge (y_2);
\node[align=left, below] at (0.1,-.9)%
{Verkko \textit{X}};
\node[align=right, below] at (7.5,-.9)%
{Verkko \textit{Y}};
\end{tikzpicture}
\end{center}
Kierros 1:
\begin{itemize}
  \item Oletetaan, että Pelaaja I valitsee solmun $x_2$. Tällöin meillä on $a_1 \coloneqq x_2, b_1 \coloneqq y_1$ ensimmäisen kierroksen jälkeen.
\end{itemize}
\end{frame}

\begin{frame}
\begin {center}
\begin {tikzpicture}[-latex ,auto ,node distance =4 cm and 5cm ,on grid ,
semithick ,
state/.style ={ circle ,top color =white , bottom color = processblue!20 ,
draw,processblue , text=blue , minimum width =1 cm}]
\node[state] (x_1) {$x_1$};
\node[state] (x_2) [above =of x_1] {$x_2$};
\node[state] (y_2) [above right =of x_1] {$y_2$};
\node[state] (y_1) [right =of x_1] {$y_1$};
\node[state] (y_3) [right =of y_1] {$y_3$};
\path (x_1) edge (x_2);
\path (x_2) edge (x_1);
\path (y_1) edge (y_2);
\path (y_2) edge (y_1);
\path (y_3) edge (y_1);
\path (y_1) edge (y_3);
\path (y_2) edge (y_3);
\path (y_3) edge (y_2);
\node[align=left, below] at (0.1,-.9)%
{Verkko \textit{X}};
\node[align=right, below] at (7.5,-.9)%
{Verkko \textit{Y}};
\end{tikzpicture}
\end{center}
Kierros 2:
\begin{itemize}
  \item Minkä tahansa solmun Pelaaja I valitseekin, Pelaaja II voi peilata valinnan. Oletetaan, että tällä kertaa Pelaaja I valitsee verkosta $Y$.
\end{itemize}
\end{frame}

\begin{frame}
\begin {center}
\begin {tikzpicture}[-latex ,auto ,node distance =4 cm and 5cm ,on grid ,
semithick ,
state/.style ={ circle ,top color =white , bottom color = processblue!20 ,
draw,processblue , text=blue , minimum width =1 cm}]
\node[state] (x_1) {$x_1$};
\node[state] (x_2) [above =of x_1] {$x_2$};
\node[state] (y_2) [above right =of x_1] {$y_2$};
\node[state] (y_1) [right =of x_1] {$y_1$};
\node[state] (y_3) [right =of y_1] {$y_3$};
\path (x_1) edge (x_2);
\path (x_2) edge (x_1);
\path (y_1) edge (y_2);
\path (y_2) edge (y_1);
\path (y_3) edge (y_1);
\path (y_1) edge (y_3);
\path (y_2) edge (y_3);
\path (y_3) edge (y_2);
\node[align=left, below] at (0.1,-.9)%
{Verkko \textit{X}};
\node[align=right, below] at (7.5,-.9)%
{Verkko \textit{Y}};
\end{tikzpicture}
\end{center}
Kierros 2:
\begin{itemize}
\item Jos Pelaaja I valitsee solmun $y_1$, eli saman solmun kuin $b_1 = y_1$, on Pelaajan II valittava vastinpariksi Pelaajan I ensimmäisen kierroksen valinta $x_2$.
\end{itemize}
\end{frame}

\begin{frame}
\begin {center}
\begin {tikzpicture}[-latex ,auto ,node distance =4 cm and 5cm ,on grid ,
semithick ,
state/.style ={ circle ,top color =white , bottom color = processblue!20 ,
draw,processblue , text=blue , minimum width =1 cm}]
\node[state] (x_1) {$x_1$};
\node[state] (x_2) [above =of x_1] {$x_2$};
\node[state] (y_2) [above right =of x_1] {$y_2$};
\node[state] (y_1) [right =of x_1] {$y_1$};
\node[state] (y_3) [right =of y_1] {$y_3$};
\path (x_1) edge (x_2);
\path (x_2) edge (x_1);
\path (y_1) edge (y_2);
\path (y_2) edge (y_1);
\path (y_3) edge (y_1);
\path (y_1) edge (y_3);
\path (y_2) edge (y_3);
\path (y_3) edge (y_2);
\node[align=left, below] at (0.1,-.9)%
{Verkko \textit{X}};
\node[align=right, below] at (7.5,-.9)%
{Verkko \textit{Y}};
\end{tikzpicture}
\end{center}
Kierros 2:
\begin{itemize}
\item Jos Pelaaja I taas valitsee solmun $y_2$ tai $y_3$, eli jommankumman solmun $b_1 = y_1$ naapureista, Pelaajan II täytyy valita vastinpariksi solmun $a_1 = x_2$ naapuri.
\end{itemize}
\end{frame}

\begin{frame}
\begin {center}
\begin {tikzpicture}[-latex ,auto ,node distance =4 cm and 5cm ,on grid ,
semithick ,
state/.style ={ circle ,top color =white , bottom color = processblue!20 ,
draw,processblue , text=blue , minimum width =1 cm}]
\node[state] (x_1) {$x_1$};
\node[state] (x_2) [above =of x_1] {$x_2$};
\node[state] (y_2) [above right =of x_1] {$y_2$};
\node[state] (y_1) [right =of x_1] {$y_1$};
\node[state] (y_3) [right =of y_1] {$y_3$};
\path (x_1) edge (x_2);
\path (x_2) edge (x_1);
\path (y_1) edge (y_2);
\path (y_2) edge (y_1);
\path (y_3) edge (y_1);
\path (y_1) edge (y_3);
\path (y_2) edge (y_3);
\path (y_3) edge (y_2);
\node[align=left, below] at (0.1,-.9)%
{Verkko \textit{X}};
\node[align=right, below] at (7.5,-.9)%
{Verkko \textit{Y}};
\end{tikzpicture}
\end{center}
Kierros 2:
\begin{itemize}
\item Toisen kierroksen jälkeen tilanne on $a_2 \coloneqq x_2, b_2 \coloneqq y_1$ tai $a_2 \coloneqq x_1, b_2 \coloneqq y_2/y_3$.
\end{itemize}
\end{frame}

\begin{frame}
\begin {center}
\begin {tikzpicture}[-latex ,auto ,node distance =4 cm and 5cm ,on grid ,
semithick ,
state/.style ={ circle ,top color =white , bottom color = processblue!20 ,
draw,processblue , text=blue , minimum width =1 cm}]
\node[state] (x_1) {$x_1$};
\node[state] (x_2) [above =of x_1] {$x_2$};
\node[state] (y_2) [above right =of x_1] {$y_2$};
\node[state] (y_1) [right =of x_1] {$y_1$};
\node[state] (y_3) [right =of y_1] {$y_3$};
\path (x_1) edge (x_2);
\path (x_2) edge (x_1);
\path (y_1) edge (y_2);
\path (y_2) edge (y_1);
\path (y_3) edge (y_1);
\path (y_1) edge (y_3);
\path (y_2) edge (y_3);
\path (y_3) edge (y_2);
\node[align=left, below] at (0.1,-.9)%
{Verkko \textit{X}};
\node[align=right, below] at (7.5,-.9)%
{Verkko \textit{Y}};
\end{tikzpicture}
\end{center}
Kierros 2:
\begin{itemize}
\item Pelaaja II voittaa, koska kuvaus $f$ verkolta $A$ verkolle $B$, $f(a_i) = b_i, i = 1, 2$ säilyttää naapuruussuhteet, eli $f$ on osittaisisomorfismi.
\end{itemize}
\end{frame}

\begin{frame}
Metodologia teoreema: Ei ole olemassa propositiologiikanlausetta joka ilmaisee ominaisuuden $P$, jos ja vain jos, kaikilla $n \in \mathbb{Z}_+$, on olemassa mallit $A_n$ ja $B_n$, joille pätee:
\begin{itemize}
\item<2-> Ominaisuus $P$ on totta $A_n$:ssä.
\item<3-> Ominaisuus $P$ on epätotta $B_n$:ssä.
\item<4-> $A \sim_n B$, eli Pelaaja II voittaa $n$-kierroksisen EF-pelin $A$:lla ja $B$:llä.
\end{itemize}
\end{frame}

\begin{frame}
\frametitle{Esimerkki määrittelemättömyystuloksesta}
Olkoon $A$ malli, joka sisältää vain vakioita. \pause Boolean kysely: onko $A$:ssa parillinen määrä alkioita? \pause \\ $A$:n ja $B$:n konstruktio on tällöin: $|A_n| \coloneqq \{a_1, \ldots, a_n\}$ ja $|B_n| \coloneqq \{b_1, \ldots, b_n+1\}$, mielivaltaisella $n \in \mathbb{Z}_+$.
\end{frame}

\begin{frame}
\frametitle{Esimerkki jatkuu}
Konstruoidaan voittostrategia Pelaajalle II \pause Jos Pelaaja I valitsee mielivaltaisella kierroksella $i$ alkion $a_i$, niin Pelaaja II yksinkertaisesti valitsee alkion $b_i$. \pause Eli Boolean kysely onko mallissa $A$ parillinen määrä alkioita ei ole määriteltävissä propositiologiikan lauseeksi metodologia teoreeman nojalla, eikä siten myöskään relaatioalgebran kyselyksi.
\end{frame}

\end{document}