% --- Template for thesis / report with tktltiki2 class ---
% 
% last updated 2013/02/15 for tkltiki2 v1.02

\documentclass[finnish]{tktltiki2}

% tktltiki2 automatically loads babel, so you can simply
% give the language parameter (e.g. finnish, swedish, english, british) as
% a parameter for the class: \documentclass[finnish]{tktltiki2}.
% The information on title and abstract is generated automatically depending on
% the language, see below if you need to change any of these manually.
% 
% Class options:
% - grading                 -- Print labels for grading information on the front page.
% - disablelastpagecounter  -- Disables the automatic generation of page number information
%                              in the abstract. See also \numberofpagesinformation{} command below.
%
% The class also respects the following options of article class:
%   10pt, 11pt, 12pt, final, draft, oneside, twoside,
%   openright, openany, onecolumn, twocolumn, leqno, fleqn
%
% The default font size is 11pt. The paper size used is A4, other sizes are not supported.
%
% rubber: module pdftex

% --- General packages ---

\usepackage[utf8]{inputenc}
\usepackage[T1]{fontenc}
\usepackage{lmodern}
\usepackage{microtype}
\usepackage{amsfonts,amsmath,amssymb,amsthm,booktabs,color,enumitem,graphicx}
\usepackage[pdftex,hidelinks]{hyperref}

% Automatically set the PDF metadata fields
\makeatletter
\AtBeginDocument{\hypersetup{pdftitle = {\@title}, pdfauthor = {\@author}}}
\makeatother

% --- Language-related settings ---
%
% these should be modified according to your language

% babelbib for non-english bibliography using bibtex
\usepackage[fixlanguage]{babelbib}
\selectbiblanguage{finnish}

% add bibliography to the table of contents
\usepackage[nottoc]{tocbibind}
% tocbibind renames the bibliography, use the following to change it back
\settocbibname{Lähteet}

% --- Theorem environment definitions ---

\newtheorem{lau}{Lause}
\newtheorem{lem}[lau]{Lemma}
\newtheorem{kor}[lau]{Korollaari}

\theoremstyle{definition}
\newtheorem{maar}[lau]{Määritelmä}
\newtheorem{ong}{Ongelma}
\newtheorem{alg}[lau]{Algoritmi}
\newtheorem{esim}[lau]{Esimerkki}

\theoremstyle{remark}
\newtheorem*{huom}{Huomautus}

\newenvironment{tod}{\paragraph{Todistus:}}{\hfill$\square$}

% --- tktltiki2 options ---
%
% The following commands define the information used to generate title and
% abstract pages. The following entries should be always specified:

\title{Ehrenfeucht--Fraïssé-peleistä}
\author{Pauli Niva}
\date{\today}
\level{Kandidaatintutkielma}
\abstract{Tämä kirjallisuuskatsaus esittelee Ehrenfeucht--Fraïssé-pelin, sen ominaisuuksia sekä sitä, miten ja missä sitä käytetään. Ehrenfeucht--Fraïssé-peli on malliteorian työkalu, jonka avulla voidaan määritellä, toteuttavatko kaksi matemaattista rakennelmaa samat predikaattilogiikan lauseet eli ovatko rakennelmat elementaarisesti ekvivalentit. Malliteoria on matemaattisen logiikan ja laskettavuuden teorian osa-alue, joka tutkii matemaattisia rakennelmia. Näitä rakennelmia kutsutaan myös struktuureiksi tai malleiksi.

Pelin pääsovellusalue on todistuksissa, joissa osoitetaan, että jokin määrätty mallin ominaisuus ei ole ilmaistavissa predikaattilogiikan kielellä. Ehrenfeucht--Fraïssé-pelistä on kehitetty monia variaatioita eri tarpeisiin ja eri logiikoille.

Ehrenfeucht--Fraïssé pelit ovat erityisen tärkeitä äärellisten mallien teoriassa ja sen sovelluksissa tietojenkäsittelytieteessä, koska Ehrenfeucht--Fraïssé-pelit ovat yksi harvoista malliteorian tekniikoista, jotka toimivat rajoituttaessa äärettömästä äärelliseen. Monet muut yleisesti käytetyt tekniikat, kuten kompaktisuusteoreema, eivät toimi äärellisillä malleilla.}

% The following can be used to specify keywords and classification of the paper:

\keywords{Ehrenfeucht--Fraïssé-peli, äärellisten mallien teoria}

% classification according to ACM Computing Classification System (http://www.acm.org/about/class/)
% This is probably mostly relevant for computer scientists
% uncomment the following; contents of \classification will be printed under the abstract with a title
% "ACM Computing Classification System (CCS):"
\classification{Theory of computation $\rightarrow$ Finite Model Theory}

% If the automatic page number counting is not working as desired in your case,
% uncomment the following to manually set the number of pages displayed in the abstract page:
%
% \numberofpagesinformation{16 sivua + 10 sivua liitteissä}
%
% If you are not a computer scientist, you will want to uncomment the following by hand and specify
% your department, faculty and subject by hand:
%
% \faculty{Matemaattis-luonnontieteellinen}
% \department{Tietojenkäsittelytieteen laitos}
% \subject{Tietojenkäsittelytiede}
%
% If you are not from the University of Helsinki, then you will most likely want to set these also:
%
% \university{Helsingin Yliopisto}
% \universitylong{HELSINGIN YLIOPISTO --- HELSINGFORS UNIVERSITET --- UNIVERSITY OF HELSINKI} % displayed on the top of the abstract page
% \city{Helsinki}
%

\usepackage{mathtools}
\usepackage {tikz}
\usetikzlibrary {positioning}
%\usepackage {xcolor}
\definecolor {processblue}{cmyk}{0.96,0,0,0}

\begin{document}

% --- Front matter ---

\frontmatter      % roman page numbering for front matter

\maketitle        % title page
\makeabstract     % abstract page

\tableofcontents  % table of contents

% --- Main matter ---

\mainmatter       % clear page, start arabic page numbering

\section{Johdanto}
%Vuonna 1713 James Waldegrave kävi ensimmäisen tunnetun peliteoreettisen keskustelun
%kirjoittamassaan kirjeessä. Kirjeessään hän esittää minmax-strategia ratkaisun le Her -korttipelin kahden pelaajan versioon. Kuitenkin vasta Antoine Augustin Courtnot esitteli ensimmäisen yleisen peliteoreettisen analyysin vuonna 1838. Tässä työssään hän tarkastelee duopolia ja esittelee ratkaisun, joka on rajoitettu versio Nash-tasapainosta.

%Peliteorian katsotaan yleisesti kuitenkin syntyneen omana alanaan kun John von Neumann julkaisi vuonna 1928 sarjan artikkeleita. Von Neumannin työ peliteorian saralla kulminoitui vuonna 1944 kirjaan \textit{Theory of Games and Economic Behavior}, jonka hän kirjoitti yhdessä Oscar Morgensternin kanssa. Kirjassa esitellään menetelmä optimaalisen ratkaisun löytämiseksi kahden henkilön nollasummapeleissä. Logiikkaan pelit ilmestyivät 1950 -luvulla, kun Leon Henkin ehdotti pelien käyttöä antamaan äärettömille kielille semantiikoita.
Tässä tutkielmassa tarkastellaan Ehrenfeucht--Fraïssé-pelejä, joita sovelletaan logiikan määrittelemättömyystulosten todistamisessa ja tietojenkäsittelytieteessä esimerkiksi tietokantakielien ilmaisuvoiman mittaamisessa tai verkkojen tutkimisessa. Alunperin Ehrenfeucht--Fraïssé-peli määriteltiin ensimmäisen kertaluvun predikaattilogiikalle, mutta tästä pelistä kehitettiin nopeasti erilaisia variaatioita monille muille logiikoille, kuten esimerkiksi kiintopistelogiikalle (fixpoint logic) \cite{Bos93} ja lineaariselle temporaalilogiikalle (linear temporal logic) \cite{Ete96}.

Ensimmäisen kerran \textit{elementaarisen ekvivalenssin} käsite eli se, että täsmälleen samat ensimmäisen kertaluvun predikaattilogiikan lauseet ovat tosia \textit{malliteoreettisissa struktuureissa} $A$ ja $B$, esiintyy kirjallisuudessa Alfred Tarskin artikkelissa Grundzüge der Systemenkalküls 1 vuodelta 1935 \cite{Tar35}. Roland Fraïssé käytti väitöskirjatyössään \cite{Fra54} vuonna 1954 \textit{äärellistä isomorfismia} osoittaakseen, että kaksi malliteoreettista struktuuria ovat elementaarisesti ekvivalentit. Andrzej Ehrenfeucht muokkasi tästä Fraïssén menetelmästä peliteoreettisen version, joka julkaistiin vuonna 1961 Fundamenta Mathematicae:ssa \cite{Ehr61}. Nykyisin nämä pelit tunnetaan nimeltä \textit{Ehrenfeucht--Fraïssé-pelit} (jatkossa EF-pelit), joskus niitä kutsutaan myös edestakaisin-peleiksi.

Äärellinen isomorfismi eli jono \textit{osittaisisomorfismien} joukkoja siis karakterisoi elementaarisen ekvivalenssin. Tässä ideana on, että osittaisisomorfismeja tutkitaan yksi kerrallaan ja katsotaan, kuinka niitä voisi laajentaa aina yhä suuremmille osittaisisomorfismeille samalla muodostaen näistä laajennuksista joukkoja ja joukoista tarvittaessa jonoja.

Tämän tutkielman tavoitteena on esitellä täsmällisesti, mutta samalla kuitenkin havainnollisesti EF-peliä ja sen hyödyllisyyttä matemaattisen logiikan ja tietojenkäsittelytieteen saralla. Tässä työssä esitellään joitain logiikan peruskäsitteitä, mutta työn seuraaminen kuitenkin edellyttää lukijalta yliopistotasoisen matematiikan perusteiden hallintaa ja joitain logiikan peruskäsitteiden tuntemista. Lukijan oletetaan esimerkiksi tuntevan joukon ja kuvauksien käsitteet.

Luvussa kaksi esitellään lyhyesti peruskäsitteistö, kuten relaatiot, isomorfismi, kielet, mallit, elementaarinen ekvivalenssi ja sen algebrallinen karakterisointi. Kolmannessa luvussa esitellään itse EF-peli, sen kulku ja voittokriteerit ja -strategia sekä todistetaan, että toisen pelaajan voittostrategian avulla saadaan mallit jaettua ekvivalenssiluokkiin. Lisäksi kolmannessa luvussa tarkastellaan EF-pelin ominaisuuksia. Neljännessä luvussa esitellään EF-pelin sovellusaloja ja muutamia konkreettisia esimerkkejä siitä, miten EF-peliä sovelletaan. Luvussa viisi vedetään yhteen tärkeimmät havainnot ja johtopäätökset.

\section{Peruskäsitteitä}
Tässä luvussa esitellään joitakin \textit{ensimmäisen kertaluvun predikaattilogiikan} peruskäsitteitä. Alaluku $2.1$ käsittelee relaatioita. Alaluvussa $2.2$ määritellään kieli ja sen alakäsitteet aakkosto, termit ja atomikaavat. Alaluvussa $2.3$ puolestaan määritellään mallin sekä alimallin käsitteet. Alaluku $2.4$ keskittyy isomorfiaan ja osittaisisomorfiaan. Lisäksi alaluvussa esitetään Tarskin totuusmääritelmä. Alaluvussa $2.5$ määritellään elementaarinen ekvivalenssi ja esitellään isomorfian ja elementaarisen ekvivalenssin keskinäistä suhdetta. Peruskäsitteiden määrittelyssä seuraan karkeasti Wilfrid Hodgesia \cite{Hod97}.

\subsection{Relaatiot}
Olkoon $X$ jokin joukko. Joukon $X$ $n$\textit{-kertainen karteesinen tulo} tarkoittaa kaikkien joukon $X$ alkioiden $n$-pituisten jonojen joukkoa. Tätä merkitään $X^n$ tai vaihtoehtoisesti $X \times X \times \ldots \times X$, jossa $X$ esiintyy $n$ kertaa. Esimerkiksi joukko $\mathbb{R}^2$ on järjestettyjen reaalilukuparien joukko. Sen geometrinen vastine on taso. $\mathbb{R}^3$ on järjestettyjen reaalilukukolmikoiden joukko. Sen geometrinen vastine on kolmiulotteinen avaruus. 

\begin{maar}[Kaksipaikkainen relaatio]
Joukon $X$ \textit{kaksipaikkainen relaatio} $R$ on mikä tahansa joukko joukon $X$ alkioista muodostettuja pareja $(x, y)$, joiden molemmat alkiot ovat joukossa $X$, eli $R \subset X^2$.  Jos $(x, y) \in R$, sanotaan, että $x$ on $y$:n kanssa \textit{relaatiossa} $R$. Joukon $X$ kaksipaikkainen relaatio $R$ on:
\begin{itemize}
\item \textit{refleksiivinen}, jos $(x, x) \in R$, kaikilla $x \in X$.
\item \textit{irrefleksiivinen}, jos $(x, x) \notin R$, kaikilla $x \in X$.
\item \textit{symmetrinen}, jos $(x, y) \in R$, aina kun $(y, x) \in R$.
\item \textit{antisymmetrinen}, jos seuraava ehto toteutuu: jos $(x, y) \in R$ ja $(y, x) \in R$, niin $x = y$.
\item \textit{transitiivinen}, jos seuraava ehto toteutuu: jos $(x, y) \in R$ ja $(y, z) \in R$, niin $(x, z) \in R$.
\item \textit{vertailullinen}, jos $(x, y) \in R$ tai $(y, x) \in R$, kaikilla $x, y \in X$, $x \neq y$.
\end{itemize}
\end{maar}

\subsection{Kielet}
Tässä alaluvussa määritellään aakkosto, termit, atomikaavat sekä ensimmäisen kertaluvun predikaattilogiikan kieli. Tarkemmin sanottuna, ei ole olemassa vain yhtä predikaattilogiikan kieltä, vaan jokaista aakkostoa kohden on oma kielensä, jolla voidaan puhua sen aakkoston malleista.

\begin{maar}[Aakkosto]
Olkoot $l, m$ ja $n$ kardinaalilukuja. \textit{Aakkosto} on joukko $L = \{R_i \mid i < l\} \cup \{c_i \mid i < m\} \cup \{f_i \mid i < n\}$, joka sisältää $l$ relaatiosymbolia $R_i$, $m$ vakiosymbolia $c_i$ ja $n$ funktiosymbolia $f_i$.
\end{maar}
Jokaiseen relaatiosymboliin $R$ liittyy \textit{paikkaluku} $\#R$ ilmaisemaan sitä, kuinka monipaikkainen kyseinen relaatio on. Samoin jokaiseen funktiosymboliin liittyy paikkaluku $\#f$ ilmaisemaan sitä, kuinka monipaikkainen funktio on kyseessä. Jos jokin kardinaaliluvuista on $0$, niin tällöin tätä vastaavia symboleja ei ole aakkostossa. Aakkostoa kutsutaan \textit{relationaaliseksi}, jos se ei sisällä funktiosymboleja tai vakiosymboleja.

\begin{maar}[Termit]
Olkoon $R$ joukko relaatioita, $C$ joukko vakioita ja $F$ joukko funktioita, jotka muodostavat aakkoston $L$. Olkoon $X$ joukko muuttujia. \textit{Termien} joukko $T$ yli aakkoston $L$ on joukko äärellisiä merkkijonoja, joka määritellään seuraavasti:
\begin{itemize}
\item Jos $x \in X$, niin $x \in T$.
\item Jos $c \in C$, niin $c \in T$.
\item Jos $f \in F$, $\#f = n$, $n\in \mathbb{Z}_+$ ja $t_1, \ldots, t_n \in T$, niin $f(t_1, \ldots, t_n) \in T$.
\end{itemize}
\end{maar}

\begin{maar}[Atomikaavat]
Olkoon $R$ joukko relaatioita, $C$ joukko vakioita ja $F$ joukko funktioita, jotka muodostavat aakkoston $L$. Olkoon $T$ termien joukko yli aakkoston $L$. \textit{Atomikaavojen} joukko $A$ on joukko äärellisiä merkkijonoja, joka määritellään seuraavasti:
\begin{itemize}
\item Jos $s, t \in T$, niin $s = t \in A$. Toisin sanoen $s = t$ on atomikaava.
\item Jos $r \in R$, $\#r = n$, $n\in \mathbb{Z}_+$ ja $t_1, \ldots, t_n \in T$, niin $r(t_1, \ldots, t_n) \in A$.
\end{itemize}
\end{maar}

\begin{maar}[Kieli]
Olkoon $A$ atomikaavojen joukko ja olkoon $X$ muuttujien joukko. \textit{Kieli} $K$ yli aakkoston $L$ on kokoelma merkkijonoja, jotka muodostetaan rekursiivisesti atomikaavoista seuraavanlaisesti:
\begin{itemize}
\item Kaikki atomikaavat kuuluvat kieleen $K$ eli $A \subset K$.
\item Jos $\psi$ ja $\varphi \in K$, niin $\neg \varphi \in K$, $(\psi \lor \varphi) \in K$, $(\psi \land \varphi) \in K$, $(\psi \to \varphi) \in K$, sekä $(\psi \leftrightarrow \varphi) \in K$.
\item Jos $\varphi \in K$ ja $x \in X$, niin $\forall x(\varphi) \in K$ ja $\exists x(\varphi) \in K$
\end{itemize}
\end{maar}
Kieli $K$ on ensimmäisen kertaluvun predikaattilogiikan kieli, joka sisältää merkkijonoja. Näitä merkkijonoja kutsutaan kirjallisuudessa usein myös \textit{kaavoiksi}. Kaavojen \textit{alikaavoja} ovat kaikki kaavan osat, jotka itsekin ovat kaavoja. 

\subsection{Mallit}
Tässä alaluvussa määritellään, mikä on \textit{malli} eli \textit{struktuuri}. Karkeasti ottaen se on joukko, jolla on jonkinlainen rakenne ja joka koostuu relaatioista, vakioista ja funktioista. Malli ja struktuuri ovat synonyymeja. Näitä kahta sanaa käytetään rinnakkain konteksista riippuen sen mukaan, kumpi soveltuu kyseiseen tilanteeseen. Sanaa ``malli'' käytetään yleensä kontekstissa ``Olkoon $M$ malli kaavalle $\varphi$'', mikä tarkoittaa samaa kuin ``Olkoon $M$ struktuuri siten, että $M \models \varphi$'' Nämä konseptit määritellään myöhemmin tässä kappaleessa.

\begin{maar}[Malli] Olkoon $L$ aakkosto ja olkoon $M$ epätyhjä joukko. Tällöin $L$\textit{-malli} koostuu seuraavista:
\begin{itemize}
\item Joukosta $M$.
\item Relaatioista $R^M \subset M^n$, jokaiselle relaatiosymbolille $R \in L$, $\#R = n$, missä $n\in \mathbb{Z}_+$.
\item Vakioista $c^M \in M$, jokaiselle vakiosymbolille $c \in L$.
\item Funktioista $f^M: M^m \rightarrow M$, jokaiselle funktiosymbolille $f \in L$, $\#f = m$, missä $m\in \mathbb{Z}_+$.
\end{itemize}

Relaatiota $R^M$ sanotaan relaatiosymbolin $R$ \textit{tulkinnaksi mallissa} $M$, funktiota $f^M$ sanotaan funktiosymbolin $f$ \textit{tulkinnaksi mallissa} $M$, ja alkiota $c^M$ kutsutaan vakiosymbolin $c$ \textit{tulkinnaksi mallissa} $M$.
\end{maar}

Malli siis antaa aakkoston $L$ symboleille \textit{semantiikan} eli merkityksen sekä kontekstin, jossa formaalin kielen lauseet voivat olla tosia tai epätosia. Samaa merkintää $M$ käytetään sekä mallista kokonaisuutena että mallin \textit{universumista} eli mallin alkioiden joukosta.

Relaatiosymbolin $R$ ja sen tulkinnan $R^M$ välistä eroa voidaan havainnollistaa esimerkiksi sanan ``tietokone'' ja tietokoneen välisellä erolla. Sana ``tietokone'' on suomen kieltä, joka koostuu yhdeksästä merkistä ja sitä voidaan käyttää muodostettaessa suomenkielisiä lauseita. Tietokone taas on fyysinen laite, joka käsittelee tietoa ohjelmointinsa mukaisesti, eikä sitä voida käyttää suomen kielen lauseiden osana. Lause ``Tietokoneeni on Mac'' on totta tai epätotta, riippuen siitä, mihin nimenomaiseen tietokoneeseen sana ``tietokone'' viittaa.

\begin{maar}[Alimalli]
Oletaan, että $A$ on $L$-malli. $A$:n \textit{alimalli} on sellainen $L$-malli $B$, että
\begin{itemize}
\item $B \subset A$.
\item jos $R \in L$ on relaatiosymboli ja $\#R = n \in \mathbb{Z}_+$, niin tällöin $R^B = R^A \cap B^n$.
\item jos $c \in L$ on vakiosymboli, niin tällöin $c^B = c^A$ ja $c^B \in B$.
\item jos $f \in L$ on funktiosymboli ja $\#f = n \in \mathbb{Z}_+$, niin tällöin $f^A(B^n) \subset B$ ja $f^B = f^A \restriction B^n$ eli $f^B$ on $f^A$:n rajoittuma osajoukkoon $B^n$. Siis $B$ on suljettu $f^A$:n suhteen.
\end{itemize}
\end{maar}

Olkoon $A$ malli ja $B \subset A$. Tällöin $\langle B \rangle$ on pienin $A$:n alimalli, joka sisältää joukon $B$. 

\subsection{Isomorfia}
Mallien kohdalla puhuttiin ``jonkinlaisesta rakenteesta'' eli struktuurista. Mallin objekteille annettiin nimiä, symboleja ja kaavoja, jotta tätä rakennetta voitiin kuvailla. Malleja, joiden rakenne on samanlainen, kutsutaan isomorfisiksi.

\begin{maar}[Isomorfismi]
Oletetaan, että $L$ on aakkosto ja $A$ sekä $B$ ovat $L$-malleja. Kuvaus $g: A \rightarrow B$ on \textit{isomorfismi} mallista $A$ mallille $B$, jos
\begin{itemize}
\item $g$ on bijektio.
\item jokaisella vakiosymbolilla $c \in L$ pätee $g(c^A) = c^B$.
\item jokaisella relaatiosymbollilla $R \in L, \#R = n$ pätee $(a_1, \ldots, a_n) \in R^A \Longleftrightarrow (g(a_1), \ldots, g(a_n)) \in R^B$.
\item jokaisella funktiosymbolilla $f \in L, \#f = m$ pätee $g(f^A(a_1, \ldots, a_m)) = f^B(g(a_1), \ldots, g(a_m))$.
\end{itemize}
Jos on olemassa isomorfinen kuvaus $A \rightarrow B$, niin sanotaan, että $A$ ja $B$ ovat isomorfiset ja tätä merkitään $A \cong B$.
\end{maar}

Isomorfia mallien välillä on refleksiivinen, symmetrinen ja transitiivinen. Jotta mallien välillä voisi olla isomorfia, mallien täytyy olla samankokoiset, sillä muuten niiden välillä ei voi olla bijektiota eikä siten isomorfiaakaan. EF-pelien kannalta tärkeä isomorfian ominaisuus on, että se säilyttää totuuden.

\begin{maar}[Osittaisisomorfismi]
Olkoon $L$ aakkosto sekä olkoot $A$ ja $B$ kumpikin $L$-malleja. Olkoon $A' \subset A$ ja $B' \subset B$. Lisäksi olkoon $f: A' \rightarrow B'$. Kuvausta $f$ kutsutaan \textit{osittaisisomorfismiksi}, jos on olemassa isomorfismi $g: \langle A' \rangle \rightarrow \langle B' \rangle$, jolle $g \restriction A' = f$ eli kuvaus $f$ on kuvauksen $g$ rajoittuma osajoukkoon $A'$. Tästä käytetään merkintää $A \cong_p B$.
\end{maar}

Toisin kuin isomorfismissa, osittaisisomorfismissa totuus ei välttämättä säily. Joissain tilanteissa osittaisisomorfismi kuitenkin säilyttää totuuden. Erityisesti näin on relationaalisten aakkostojen tapauksessa.

Määritellään seuraavaksi, mitä keskeisellä käsitteellä \textit{totuus} tarkalleen ottaen tarkoitetaan.

\begin{maar}[Tarskin totuusmääritelmä]
Oletetaan, että $A$ on $L$-malli ja olkoon $L$ aakkosto ja $K$ kieli. Määritellään rekursiivisesti, että kielen $K$ kaava $\varphi$ on totta $A$:ssa eli $A$ toteuttaa $\varphi$:n eli $A \models \varphi$ seuraavasti:
\begin{itemize}
\item Jos $\varphi$ on kaava $s = t$ jossa $s$ ja $t$ ovat termejä, niin $A \models \varphi$ jos ja vain jos $s^A = t^A$.
\item Jos $\varphi$ on kaava $R(t_1, \ldots, t_n)$, missä $R$ on $n$-paikkainen relaatiosymboli ja $t_1, \ldots, t_n$ ovat termejä, niin $A \models \varphi$ jos ja vain jos $(t_1^A, \ldots, t_n^A) \in R$.
\item $A \models \neg \varphi$ jos ja vain jos $A \not \models \varphi$.
\item $A \models (\varphi \land \psi)$ jos ja vain jos $A \models \varphi$ ja $A \models \psi$.
\item $A \models (\varphi \lor \psi)$ jos ja vain jos $A \models \varphi$ tai $A \models \psi$.
\item $A \models (\varphi \to \psi)$ jos ja vain jos $A \not \models \varphi$ tai $A \models \psi$.
\item $A \models (\varphi \leftrightarrow \psi)$ jos ja vain jos $A \models \varphi$ ja $A \models \psi$ tai $A \not \models \varphi$ ja $A \not \models \psi$.
\item $A \models \forall x(\varphi)$ jos ja vain jos jokaisella mallin $A$ alkiolla $a
$ pätee $A \models \varphi$ kun muuttaja $x$ tulkitaan $a$:ksi.
\item $A \models \exists x(\varphi)$ jos ja vain jos löytyy jokin mallin $A$ alkio $a$ jolla $A \models \varphi$ kun muuttuja $x$ tulkitaan $a$:ksi.
\end{itemize}
\end{maar}

\subsection{Elementaarinen ekvivalenssi}
Siinä missä isomorfismi kuvailee kahden mallin rakenteellista samanlaisuutta, niin elementaarinen ekvivalenssi puolestaan vertailee malleja suhteessa käytettyyn kieleen.

\begin{maar}[Elementaarinen ekvivalenssi]
Olkoon $L$ aakkosto joka muodostaa kielen $K$. Olkoon $A$ ja $B$ kummatkin $L$-malleja. $A$:ta ja $B$:tä sanotaan \textit{elementaarisesti ekvivalenteiksi}, jos kaikilla lauseilla $S \in K$ pätee $A \models S \Longleftrightarrow B \models S$. Tätä merkitään $A \equiv B$
\end{maar}
Toisin sanoen, jos kaksi mallia ovat elementaarisesti ekvivalentit, niitä ei voi erotella toisistaan millään ensimmäisen kertaluvun predikaattilogiikan lauseella. Isomorfian ja elementaarisen ekvivalenssin määritelmistä voidaan osoittaa seuraava tärkeä tulos:
\begin{kor}
Jos $L$-mallit $A$ ja $B$ ovat isomorfiset, niin ne ovat elementaarisesti ekvivalentit.
\end{kor}

On huomattava, että tämä ei päde toisinpäin. Mallien $A$ ja $B$ välinen elementaarinen ekvivalenssi ei kerro mitään mallien isomorfisuudesta. Korollaarin 12 johdosta elementaarinen ekvivalenssi on epäkiinnostava äärellisten mallien kontekstissa, sillä ensimmäisen kertaluvun predikaattilogiikka karakterisoi äärelliset mallit isomorfiaan asti.
\begin{lau}
Jokaiselle äärelliselle mallille $A$ on olemassa ensimmäisen kertaluvun predikaattilogiikan lause $\varphi$ siten, että $B \models \varphi$, jos ja vain jos mielivaltainen malli $B$ ja malli $A$ ovat isomorfiset.
\end{lau}

Vaikka jokaiselle mallille on ensimmäisen kertaluvun predikaattilogiikan lause, joka karakterisoi kyseisen mallin, isomorfia on kuitenkin liian vahva elementaarisen ekvivalenssin ilmaisemiseen. Esimerkiksi mitkä tahansa kaksi tiheää lineaarijärjestystä ilman päätepisteitä ovat elementaarisesti ekvivalentit, mutta eivät kuitenkaan ole isomorfiset. Erityisesti lineaarijärjestykset $(\mathbb{Q}, <)$ ja $(\mathbb{R}, <)$ ovat elementaarisesti ekvivalentit, joten reaalilukujen täydellisyyttä ei voi ilmaista ensimmäisen kertaluvun predikaattilogiikan kielellä.

Ratkaisu siihen, että isomorfismi on liian vahva elementaarisen ekvivalenssin ilmaisemiseen on isomorfismin heikennys, äärellinen isomorfismi. Fraïssén kehittämä äärellinen isomorfismi karakterisoi elementaarisen ekvivalenssin. Tässä tutkielmassa ei kuitenkaan ole tarvetta erikseen määritellä äärellistä isomorfismia, koska kaikkiin käytettyihin esimerkkeihin riittää osittaisisomorfismi. Tästä syystä tässä tutkielmassa on otettu vapaus heikentää Fraïssén lausetta, siten että äärellisen isomorfismin sijaan rajoitutaan osittaisisomorfismiin. Tämä heikennys oikeutetaan sillä, että jos kaksi mallia ovat osittaisisomorfiset, niin tällöin ne ovat äärellisesti isomorfiset.

\begin{lau}[Fraïssén lauseen heikennetty versio]
Olkoon $L$ aakkosto ja olkoot $A$ ja $B$ kumpikin $L$-malleja. Tällöin $A \equiv B$ jos ja vain jos $A \cong_p B$.
\end{lau}

Fraïssén teoreema, tarkemmin sen heikennetty muoto siis antaa algebrallisen keinon käsitellä elementaarista ekvivalenssia, jonka käyttö suoraan määritelmästä olisi hyvin hankalaa ja rajallista. Toinen keino käsitellä elementaarista ekvivalenssia on Fraïssén teoreeman peliteoreettinen muotoilu EF-peli.

\section{EF-peli}
Tässä kappaleessa esitellään EF-peli, sen säännöt, strategian ja voittavan strategian käsitteet, joissa seuraan pitkälti Jouko Väänästä \cite{Vaa11} ja havainnollistetaan EF-peliä esimerkillä kahdelle verkolle.

EF-peli on kahdelle pelaajalle, joita kutsutaan nimillä Pelaaja I ja Pelaaja II. Peliä pelataan kahdella mallilla $A$ ja $B$, joilla on sama relationaalinen aakkosto. \begin{huom}
Jatkossa aakkostolla tarkoitetaan aina relationaalista aakkostoa, ellei toisin mainita.
\end{huom} Pelaaja II haluaa osoittaa, että kyseiset mallit ovat jossain määrin samankaltaiset eli että kyseisten mallien välillä on osittaisisomorfismi, kun taas Pelaaja I haluaa osoittaa, että mallit ovat erilaiset. Pelissä on äärellinen määrä vuoroja ja vuorojen määrä on alussa sovittu.

\subsection{Pelin kulku}
Pelin kulku kuvataan kirjallisuudessa lähes aina samalla tavalla. Määritellään aluksi mielivaltaisen kierroksen kulku ja kummankin pelaajan voittokriteerit.

\begin{maar}[Kierroksen kulku]
Merkitään pelattavien kierrosten määrää luvulla $k \in \mathbb{Z}_+$. EF-peliä pituudeltaan $k$-kierrosta malleilla $A$ ja $B$ merkitään $EF_k(A, B)$. Pelin $EF_k(A, B)$ mielivaltaisen kierroksen $i \in \{1, \ldots, k\}$ kulku on seuraavanlainen: 
\begin{itemize}
\item Ensin Pelaaja I valitsee toisen malleista $A$ tai $B$ sekä jonkin alkion $a_i \in A$ tai $b_i \in B$ tästä mallista.
\item Tämän jälkeen Pelaaja II valitsee malleista sen, jota Pelaaja I ei valinnut ja valitsee tästä mallista jonkin alkion.
\end{itemize}
\end{maar}

\begin{maar}[Voittokriteeri]
Olkoon $a = (a_1, \ldots, a_i)$ mallista $A$ valitut alkiot ja $b = (b_1, \ldots, b_i)$ mallista $B$ valitut alkiot mielivaltaisella kierroksella $i$. Pelaajan II voittaa jos ja vain jos jokaisella kierroksella $i \leq n$ pari $(a_i, b_i)$ määrää osittaisen isomorfismin $A \rightarrow B$ eli on olemassa kuvaus $h: A \rightarrow B$, siten että $a_i \in A \mapsto b_i \in B$ Muussa tapauksessa Pelaaja I voittaa.
\end{maar}

\textit{Strategia} on joukko sääntöjä, joiden mukaan pelaaja tekee valintansa toisen pelaajan valinnasta riippuen. EF-pelissä kummallakin pelaajalla on koko ajan tiedossa mallit, niiden rakenne ja jo tehdyt valinnat, eli peli on \textit{täydellisen informaation peli}. Strategiaa, jota seuraamalla pelaaja voittaa pelin riippumatta mitä valintoja toinen pelajaa tekee kutsutaan \textit{voittavaksi strategiaksi}. Jos Pelaaja II:lla on voittava strategia EF-pelissä $EF_k(A, B)$, niin tätä merkitään $A \sim_k B$.

\begin{lau} 
Relaatio $\sim_k$ on $L$-mallien ekvivalenssirelaatio.
\end{lau}
\begin{tod} Todistus on Jouko Väänäsen kirjassa \cite{Vaa11} esiintyvää todistusta mukaileva. Oletetaan, että $A, B$ ja $C$ ovat kaikki saman aakkoston $L$-malleja. Tällöin
\begin{itemize}
\item Refleksiivisyys: $A \sim_k A$.\\
Voittava strategia Pelaajalle II on valita aina sama alkio minkä Pelaaja I valitsi. Täten $\sim_k$ on refleksiivinen.
\item Symmetrisyys: $A \sim_k B \iff B \sim_k A$.\\
EF-pelin määritelmä ei millään tavoin tee eroa pelien $EF_k(A, B)$:n ja $EF_k(B, A)$:n välillä. Täten jos $A \sim_k B$, niin Pelaaja II voi käyttää samaa voittostrategiaa myös pelissä $EF_k(B, A)$. Jos taas $B \sim_k A$, niin Pelaaja II voi käyttää samaa voittostrategiaa pelissä $EF_k(A, B)$. Siis $\sim_k$ on symmetrinen.
\item Transitiivisuus: $A \sim_k C \land B \sim_k C \implies A \sim_k C$.\\
Todistetaan väite käsittelemällä kaikki väitteen ja implikaation pelit samaan aikaan. Peli $EF_k(A, C)$ pelataan siten, että Pelaaja II tekee valintansa pelaamalla samaan aikaa kuvitteellisia pelejä $EF_k(A, B)$ ja $EF_k(B, C)$. Oletetaan, että peli $EF_k(A, C)$ alkaa Pelaajan II valinnalla $a_1 \in A$, jolloin Pelaaja II valitsee seuraavalla strategialla:

Pelaaja II kuvittelee, että Pelaaja I valitsi alkion $a_1 \in A$ pelissä $EF_k(A, B)$, jolloin hän valitsee alkion $b_1 \in B$ pelin $EF_k(A, B)$ voittostrategian mukaisesti. Seuraavaksi Pelaaja II kuvittelee, että äskeinen valinta oli Pelaajan I valinta pelissä $EF_k(B, C)$ ja valitsee alkion $c_1 \in C$ pelin $EF_k(B, C)$ voittostrategian mukaisesti. Tämä valinta $c_1$ on Pelaajan II vastaus Pelaajan I valintaan $a_1$ pelissä $EF_k(A, C)$.

Jos Pelaaja I valitseekin alkion $c_1 \in C$, niin Pelaaja II yksinkertaisesti seuraa strategiaa toiseen suuntaan. Näin voidaan toimia, koska äsken todistimme, että voittavat strategiat ovat symmetrisiä.

Kun peliä on pelattu $k$-kierrosta, niin meillä on valinnoista muodostuneet jonot $a_1, \ldots, a_k$, $b1, \ldots, b_k$ ja $c_1, \ldots, c_k$. Oletusten nojalla on olemassa osittaisisomorfismit $f: A \cong_p B$ ja $g: B \cong_p C$. Nyt voidaan muodostaa yhdistetty kuvaus $h(f(a_i)) = g(a_i)$, siten että $a_i \mapsto c_i$ on osittaisisomorfismi, kaikilla $i \in \{1, \ldots, k\}$. Täten $A \sim_k C$, siis relaatio $\sim_k$ on transitiivinen.
\end{itemize}
Koska relaatio $\sim_k$ on refleksiivinen, symmetrinen ja transitiivinen, niin se on tällöin ekvivalenssirelaatio.
\end{tod}
\\

Kuten aikaisemmin todettiin, EF-pelistä on kehitetty monia erilaisia variaatioita, kuten esimerkiksi EF-peli verkoille. Verkoilla voidaan esittää ensimmäisen kertaluvun predikaattilogiikan kaavoja. Verkon kaaret ovat relaatioita verkon pisteiden välillä. Havainnollistetaan nyt EF-pelin ideaa yksinkertaisella esimerkillä kahden kierroksen EF-pelillä verkoille $X$ ja $Y$, sekä esitetään samalla strategia $X \sim_k Y$, eli voittava strategia Pelaajalle II.

\begin{esim}
Tässä EF-pelissä verkoille ideana on, että rakennetaan pelaajien tekemistä valinnoista kahta uutta verkkoa $A$ ja $B$, siten että verkosta $X$ valittu solmu merkitään verkon $A$ solmuksi $a_i$ ja verkosta $Y$ valittu solmu verkon $B$ solmuksi $b_i$, jossa $i$ ilmaisee millä kierroksella valinta on tapahtunut. Uudet verkot $A$ ja $B$ rakennetaan puhtaasti helpottamaan tehtyjen valintojen ja näiden relaatioiden muodostaman kokonaisuuden hahmottamista ja hallintaa.

\begin{center}
\begin{tikzpicture}[-latex ,auto ,node distance =4 cm and 5cm ,on grid ,
semithick ,
state/.style ={ circle ,top color =white , bottom color = processblue!20 ,
draw,processblue , text=blue , minimum width =1 cm}]
\node[state] (x_1) {$x_1$};
\node[state] (x_2) [above =of x_1] {$x_2$};
\node[state] (y_2) [above right =of x_1] {$y_2$};
\node[state] (y_1) [right =of x_1] {$y_1$};
\node[state] (y_3) [right =of y_1] {$y_3$};
\path (x_1) edge (x_2);
\path (x_2) edge (x_1);
\path (y_1) edge (y_2);
\path (y_2) edge (y_1);
\path (y_3) edge (y_1);
\path (y_1) edge (y_3);
\path (y_2) edge (y_3);
\path (y_3) edge (y_2);
\node[align=left, below] at (0.1,-.9)%
{Verkko \textit{X}};
\node[align=right, below] at (7.5,-.9)%
{Verkko \textit{Y}};
\end{tikzpicture}
\end{center}


Kierros 1:
\begin{itemize}
  \item Pelaaja I voi valita solmun kummasta verkosta tahansa.
  \item Jos Pelaaja I valitsee solmun verkosta $X$, Pelaaja II valitsee vastinpariksi verkosta $Y$ solmun $y_1$, muulloin Pelaaja II valitsee vastinpariksi solmun $x_1$.
  \item Oletetaan, että Pelaaja I valitsee solmun $x_2$. Tällöin meillä on $a_1 \coloneqq x_2, b_1 \coloneqq y_1$ ensimmäisen kierroksen jälkeen.
\end{itemize}

Kierros 2:
\begin{itemize}
  \item Minkä tahansa solmun Pelaaja I valitseekin, Pelaaja II voi peilata valinnan. Oletetaan, että tällä kertaa Pelaaja I valitsee verkosta $Y$.
  \item Jos Pelaaja I valitsee solmun $y_1$, eli saman solmun kuin $b_1 = y_1$, on Pelaajan II valittava vastinpariksi Pelaajan I ensimmäisen kierroksen valinta $x_2$.
  \item Jos Pelaaja I taas valitsee solmun $y_2$ tai $y_3$, eli jommankumman solmun $b_1 = y_1$ naapureista, Pelaajan II täytyy valita vastinpariksi solmun $a_1 = x_2$ naapuri.
  \item Toisen kierroksen jälkeen tilanne on $a_2 \coloneqq x_2, b_2 \coloneqq y_1$ tai $a_2 \coloneqq x_1, b_2 \coloneqq y_2/y_3$.
  \item Pelaaja II voittaa, koska kuvaus $f$ verkolta $A$ verkolle $B$, $f(a_i) = b_i, i = 1, 2$ säilyttää naapuruussuhteet, eli $f$ on osittaisisomorfismi.
\end{itemize}

Jos Pelaaja I olisi tehnyt toisellakin kierroksella valintansa verkosta $X$:
\begin{itemize}
\item Jos Pelaaja I valitsee solmun $x_1$, niin Pelaaja II valitsee solmun $y_1$.
\item Jos Pelaaja I taas valitsee solmun $x_2$, niin Pelaaja II valitsee solmun $y_2$.
\item Tällöin toisen kierroksen jälkeen tilanne olisi ollut $a_2 \coloneqq x_1, b_2 \coloneqq y_1$ tai $a_2 \coloneqq x_2, b_2 \coloneqq y_2$.
\item Pelaaja II voittaa tässäkin skenaariossa, koska kuvaus $f$ verkolta $A$ verkolle $B$, $f(a_i) = b_i, i = 1, 2$ säilyttää naapuruussuhteet, eli $f$ on osittaisisomorfismi.
\end{itemize}
\end{esim}

\subsection{EF-pelin ominaisuuksia}
EF-peleillä on monia mielenkiintoisia ominaisuuksia joista muutamia esitellään tässä alaluvussa. Aikaisemmin esiteltiin voittostrategian käsite jossa todettiin, että voittostrategian omaava pelaaja voittaa kyseisen pelin teki toinen pelaaja mitä tahansa. Tämän suora seuraus on, että vain toisella pelaajalla voi olla voittostrategia kyseisessä EF-pelissä. Tarkemmin sanottuna:
\begin{lau}
Olkoon $A$ ja $B$ malleja. Tällöin täsmälleen toisella pelaajista I ja II on voittostrategia $k$-kierroksisessa EF-pelissä $EF_k(A, B)$.
\end{lau}

Koska osittaisisomorfismin $A \rightarrow B$ pitää toteutua jokaisella kierroksella, jotta Pelaaja II voittaa EF-pelin $EF_k(A, B)$, olisi Pelaaja II voittanut myös kaikki EF-pelit $EF_r(A, B), r \leq k$. Tämä ominaisuus voidaan ilmaista voittostrategian näkökulmasta seuraavasti:
\begin{lau}
Jos Pelaajalla II on voittostrategia EF-pelissä $EF_k(A, B)$, niin Pelaajalla II on voittostrategia kaikissa peleissä $EF_r(A, B), r \leq k$.
\end{lau}

Tämä ei kuitenkaan päde peleillä $EF_t(A, B), k < t$. Esimerkiksi jos esimerkissä 16 pelattu kahden kierroksen EF-peli verkoille oltaisiin pelattu kolme kierroksisena pelinä, voittostrategia olisikin Pelaajalla I. Hän yksinkertaisesti valitsisi yksi kerrallaan kaikki verkon $Y$ solmut, jolloin kolmannella kierroksella Pelaaja II ei enää pystyisi valitsemaan solmua joka säilyttää relaatiot.

Lause 18 ei kuitenkaan päde kaikissa EF-peleissä. Esimerkiksi Vadim Kulikov ja Tapani Hyttinen ovat kehittäneet EF-pelistä version jota he kutsuvat Heikoksi EF-Peliksi, jossa Pelaajalla II on voittostrategia suurempikierroksisessa pelissä, mutta Pelaajalla I voi olla voittostrategia samoilla malleilla pelattavassa vähempikierroksisessa EF-pelissä \cite{Hyt11}.

Heikon EF-pelin ero tavalliseen EF-peliin on siinä, että Heikossa EF-pelissä osittaisisomorfismi voi olla mielivaltainen, kun tavallisessa EF-pelissä osittaisisomorfian pitää olla pelaajien tekemien valintojen määräämä. Toisin sanoen Heikossa EF-pelissä riittää, että pelin päättyessä mallien $A$ ja $B$ välillä on osittaisisomorfismi $A \rightarrow B$, mutta näin ei välttämättä täydy olla pelin aikaisempien kierroksien jälkeen.

Esitellään lopuksi Ehrenfeuchtin ja Fraïssen teoreema, eli miten EF-peli, elementaarinen ekvivalenssi ja Fraïssén teoreema nivoutuvat yhteen.
\begin{lau}[Ehrenfeucht--Fraïssé]
Olkoon $L$ aakkosto ja olkoot $A$ ja $B$ kumpikin $L$-malleja. Tällöin seuraavat ovat yhtäpitäviä kaikilla kokonaisluvuilla $k$:
\begin{itemize}
\item $A \equiv B$, toisin sanoen $A$ ja $B$ eivät ole erotettavissa toisistaan ensimmäisen kertaluvun predikaattilogiikan lauseilla.
\item $A \sim_k B$, toisin sanoen Pelaajalla II on voittava strategia $k$-kierroksisessa EF-pelissä.
\end{itemize}
\end{lau}

\section{Sovelluksia}
EF-peli on teoreettisen tietojenkäsittelyn ja äärellisten mallien teorian työkalu jota pääasiassa käytetään määriteltävyyskysymyksiin ja todistusten apuna. Äärellisten mallien teoriaa ja sen menetelmää EF-peliä voidaan soveltaa tietojenkäsittelytieteessä muun muassa verifikoinnissa. 

Alaluvussa 4.1 esitellään EF-pelin yleisiä sovellusaloja. Lisäksi esitellään metodologiateoreema sekä määritellään Boolen kysely ja esimerkin avulla havainnollistetaan miten metodologiateoreemaa käytetään sovelluksissa. Alaluvussa 4.2 käsitellään mitä ensimmäisen kertaluvun predikaattilogiikalla voi ilmaista ja mitä rajoitteita sillä on. Alaluvussa 4.3 määritellään kvanttoriaste ja esitellään miten EF-pelejä sovelletaan kongruenssiapulauseiden saamiseen. Alaluvussa 4.4 määritellään ja esitellään bisimulaatio, joka on eräänlainen heikennetty versio EF-pelistä.

\subsection{Sovellusaloja}
Kaikki äärelliset mallit voidaan koodata verkkoina, puina tai merkkijonoina \cite{Ebb99}. Tällöin niitä voidaan käyttää laskennan olioina ja siten niillä voidaan kuvata äärellistilallisia systeemejä ja tutkia näiden toiminnan oikeellisuutta. 

Yksi sovellusala on tietokantateoria, koska relationaalinen malli samaistaa tietokannan äärellisen relaationaalisen struktuurin kanssa \cite{Luo10}. Formaalin kielen kaavat voidaan siis ajatella ohjelmina, jotta niiden merkitystä struktuurissa voidaan arvioida. Ja toisinpäin, voidaan esittää jonkin laskennallisen vaativuusluokan kyselyitä jollakin formaalilla kielellä.

Muita tietojenkäsittelytieteen osa-alueita joihin EF-peliä voi soveltaa on esimerkiksi vaativuusteoria, koska äärelliset mallit tarjoavat laskennan vaativuusluokkien loogisen karakterisoinnin ja mahdollistavat vaativuusteoreettisten tulosten todistamisen tätä kautta. Esimerkiksi $\mathrm{P} \neq \mathrm{NP}$ -ongelma redusoituu kysymykseksi: onko kolmella värillä väritettävien suunnattujen verkkojen luokka määriteltävissä pienimmän kiintopisteen logiikassa \cite{Imm86}? Määriteltävyystulosten todistuksissa seuraava teoreema on keskeinen:

\begin{lau}[Metodologiateoreema]
Ei ole olemassa ensimmäisen kertaluvun predikaattilogiikan lausetta joka ilmaisee ominaisuuden $P$, jos ja vain jos, kaikilla $n \in \mathbb{Z}_+$, on olemassa mallit $A$ ja $B$, joille pätee:
\begin{itemize}
\item Ominaisuus $P$ on totta $A$:ssä.
\item Ominaisuus $P$ on epätotta $B$:ssä.
\item $A \sim_n B$, eli Pelaaja II voittaa $n$-kierroksisen EF-pelin $A$:lla ja $B$:llä.
\end{itemize}
\end{lau}

Esitetään yksinkertainen relaatioalgebran ongelma esimerkkinä siitä miten metodologia teoreemaa käytetään määärittelemättömyystuloksia todistettaessa. Tässä esimerkissä käytämme Boolen kyselyä, joten ensimmäiseksi määrittelemme tämän tarkasti.

\begin{maar}[Boolen kysely]
Olkoon $M$ malli. Tällöin \textit{Boolen kysely} $Q$ on kuvaus $Q: M \to \{0, 1\}$ joka säilyy isomorfismeissa. Siis jos $A \cong B$, niin $Q(A) = Q(B)$. Kuvauksen maalijoukon alkioista $1$ merkitsee totuusarvoa ``tosi'' ja $0$ totuusarvoa ``epätosi''.
\end{maar}

\begin{esim}
Olkoon $A$ malli, joka sisältää vain vakioita. Tutkitaan Boolen kyselyä: onko $A$:ssa parillinen määrä alkioita? Konstruoidaan mallit $A$ ja $B$ todistusta varten seuraavanlaisiksi: $A \coloneqq \{a_1, \ldots, a_n\}$ ja $B \coloneqq \{b_1, \ldots, b_{n + 1}\}$, mielivaltaisella $n \in \mathbb{Z}_+$. Eli toisessa mallissa on yksi alkio enemmän kuin toisessa. Täten toinen malleista sisältää parittoman määrän alkioita ja toinen parillisen määrän alkioita.

Voittostrategia Pelaajalle II on sellainen, että Pelaaja I:den valitessa mielivaltaisella kierroksella $i$ alkion jommasta kummasta mallista, Pelaaja II yksinkertaisesti valitsee alkion toisesta mallista. Tämä strategia toimii, koska mallissa $B$ on yksi alkio enemmän kuin mikä pelattavien kierroksien määrä on, eikä mallien alkioilla ole mitään relaatioita keskenään. Koska $A \sim_n B$ ja koska toisessa mallissa on parillinen ja toisessa pariton määrä alkioita, niin ominaisuus ``parillinen määrä alkioita'' on totta toisessa mallissa ja epätotta toisessa. Täten metodologiateoreeman nojalla Boolen kysely onko mallissa $A$ parillinen määrä alkioita ei ole määriteltävissä ensimmäisen kertaluvun predikaattilogiikan lauseeksi, eikä siten myöskään relaatioalgebran kyselyksi.
\end{esim}

Edellinen esimerkki voidaan muuttaa koskemaan verkkoja lisäämällä malleihin kaarirelaatio $R$. Mallin $A$alkiot ovat verkon solmuja siten, että solmusta $a_i$ on kaari solmuun $a_{i + 1}$ ja edelleen solmusta $a_{i + 1}$ on kaari solmuun $a_{i + 2}$ ja niin edelleen, kunnes saavutaan viimeiseen solmuun. Lisäksi viimeisestä solmusta on kaari takaisin ensimmäiseen solmuun. Vastavuoroisesti $B$ muodostetaan samalla tavalla. Eli verkoille ei ole olemassa ensimmäisen kertaluvun predikaattilogiikan lausetta jolla voisi esittää ominaisuuden, että verkko sisältää parillisen määrän solmuja.

\subsection{Ensimmäisen kertaluvun predikaattilogiikan rajoitteet}
Verkoilla on paljon ominaisuuksia jotka eivät ole ensimmäisen kertaluvun predikaattilogiikka määriteltäviä. Jos $G$ on äärellisten verkkojen luokka, niin esimerkiksi seuraavat kyselyt eivät ole ensimmäisen kertaluvun predikaattilogiikka määriteltävissä $G$:lle: transitiivinen sulkeuma, tasoverkkoisuus, Eulerilaisuus, Hamiltonilaisuus, $k$-värittyvyys, kaikilla $ k \geq 2$, asyklisyys, leikkaussolmu ja verkon yhtenäisyys.

Todistetaan esimerkin vuoksi näistä epäsuorasti verkon yhtenäisyyden määrittelemättömyys.
\begin{esim}
Oletetaan, että ensimmäisen kertaluvun predikaattilogiikan lause $\varphi$ määrittelee verkkojen yhtenäisyyden aakkostossa, joka muodostuu kaksipaikkaisesta relaatiosymbolista $E$. Olkoon $L$ jokin lineaarijärjestys. Muodostestaan lineaarijärjestyksestä $L$ suunnattu verkko $G$ seuraavanlaisesti: Määritellään ensimmäiseksi seuraajarelaatio $S$ lineaarijärjestyksestä $L$ \[S(x, y) \coloneqq (x < y) \land \forall z ((z \leq x) \lor (y \leq z))\]
Määritellään ensimmäisen kertaluvun predikaattilogiikan kaava $\psi(x, y)$ siten, että $\psi(x, y)$ on toteutuva jos ja vain jos jokin seuraavista on totta:
\begin{itemize}
\item $\exists z (S(x, z) \land S(z, y)$, toisin sanoen $y$ on $x$:n seuraajan seuraaja.
\item $\forall u (y \leq u) \land (\exists z(S(x, z) \land \forall u(u \leq z)))$, toisin sanoen $x$ on viimeisen alkion edeltäjä ja $y$ on ensimmäinen alkio.
\item $\forall u (u \leq x) \land (\exists z(S(z, y) \land \forall u(z \leq u)))$, toisin sanoen $x$ on viimeinen alkio ja $y$ on ensimmäisen alkion seuraaja.
\end{itemize}
Ensimmäisen kertaluvun predikaattilogiikan kaavan $\psi$ määräämä suunnattu verkko $G$ lineaarijärjestyksen $L$ alkioista on yhtenäinen, tarkemmin sanoen se muodostuu yhdestä syklistä, jos sen solmujen määrä on pariton. Seuraavat kaksi kuvaa havainnolistavat tätä:

\begin{huom}
Verkot on numeroitu vain havainnollistamisen helpottamiseksi.
\end{huom}

\begin{center}
\begin{tikzpicture}[-latex ,auto ,node distance =2 cm and 2cm ,on grid ,
semithick ,
state/.style ={ circle ,top color =white , bottom color = processblue!20 ,
draw,processblue , text=blue , minimum width =1 cm}]
\node[state] (x_1) {1};
\node[state] (x_2) [right =of x_1] {2};
\node[state] (x_3) [right =of x_2] {3};
\node[state] (x_4) [right =of x_3] {4};
\node[state] (x_5) [right =of x_4] {5};
\path (x_1) edge [bend left] node [right] {} (x_3);
\path (x_3) edge [bend left] node [right] {} (x_5);
\path (x_5) edge [bend left] node [left]  {} (x_2);
\path (x_2) edge [bend left] node [right] {} (x_4);
\path (x_4) edge [bend left] node [left] {} (x_1);
\node[align=center, below] at (4.0,-1.3)%
{Verkko $G$ jossa on pariton määrä solmuja (lineaarijärjestystä korostaen)};
\end{tikzpicture}
\end{center}

\begin{center}
\begin{tikzpicture}[-latex ,auto ,node distance =2 cm and 2cm ,on grid ,
semithick ,
state/.style ={ circle ,top color =white , bottom color = processblue!20 ,
draw,processblue , text=blue , minimum width =1 cm}]
\node[state] at (4.0, 0) (x_1) {1};
\node[state] (x_2) [below left =of x_1] {2};
\node[state] (x_3) [right =of x_1] {3};
\node[state] (x_4) [left =of x_1] {4};
\node[state] (x_5) [below =of x_3] {5};
\path (x_1) edge (x_3);
\path (x_3) edge (x_5);
\path (x_5) edge (x_2);
\path (x_2) edge (x_4);
\path (x_4) edge (x_1);
\node[align=center, below] at (4.0,-2.6)%
{Verkko $G$ jossa on pariton määrä solmuja (syklisyyttä korostaen)};
\end{tikzpicture}
\end{center}

Jos solmujen määrä on parillinen niin verkko ei ole yhtenäinen, tarkemmin sanoen se koostuu kahdesta erillisestä syklistä. Havainnollistetaan tätä kahdella kuvalla:

\begin{center}
\begin{tikzpicture}[-latex ,auto ,node distance =2 cm and 2cm ,on grid ,
semithick ,
state/.style ={ circle ,top color =white , bottom color = processblue!20 ,
draw,processblue , text=blue , minimum width =1 cm}]
\node[state] (x_1) {1};
\node[state] (x_2) [right =of x_1] {2};
\node[state] (x_3) [right =of x_2] {3};
\node[state] (x_4) [right =of x_3] {4};
\node[state] (x_5) [right =of x_4] {5};
\node[state] (x_6) [right =of x_5] {6};
\path (x_1) edge [bend left] node [right] {} (x_3);
\path (x_3) edge [bend left] node [right] {} (x_5);
\path (x_5) edge [bend left] node [left]  {} (x_1);
\path (x_2) edge [bend left] node [right] {} (x_4);
\path (x_4) edge [bend left] node [right] {} (x_6);
\path (x_6) edge [bend left] node [left] {} (x_2);
\node[align=center, below] at (5.0,-1.6)%
{Verkko $G$ jossa on parillinen määrä solmuja (lineaarijärjestystä korostaen)};
\end{tikzpicture}
\end{center}

\begin{center}
\begin{tikzpicture}[-latex ,auto ,node distance =2 cm and 2cm ,on grid ,
semithick ,
state/.style ={ circle ,top color =white , bottom color = processblue!20 ,
draw,processblue , text=blue , minimum width =1 cm}]
\node[state] at (1.0, 0) (x_1) {1};
\node[state] at (7.0, 0) (x_2) {2};
\node[state] (x_3) [below right =of x_1] {3};
\node[state] (x_4) [below right =of x_2] {4};
\node[state] (x_5) [below left =of x_1] {5};
\node[state] (x_6) [below left =of x_2] {6};
\path (x_1) edge (x_3);
\path (x_3) edge (x_5);
\path (x_5) edge (x_1);
\path (x_2) edge (x_4);
\path (x_4) edge (x_6);
\path (x_6) edge (x_2);
\node[align=center, below] at (4.0,-2.6)%
{Verkko $G$ jossa on parillinen määrä solmuja (syklisyyttä korostaen)};
\end{tikzpicture}
\end{center}

Sijoittamalla kaavan $\psi$ relaatiosymbolin $E$ ilmentymien tilalle lauseessa $\neg \varphi$ voidaan nyt testata verkon $G$ parillisuutta. Edellisen esimerkin Boolen kyselyn ja sen yleistämisen verkoille perusteella tiedetään, ettei kyseinen testaus ole mahdollista, joten päädymme ristiriitaan. Täten ei ole olemassa ensimmäisen kertaluvun predikaattilogiikan lausetta $\varphi$ joka määrittelisi verkkojen yhtenäisyyden.
\end{esim}

Näiden lisäksi on hyvin monia muitakin verkkojen ominaisuuksia, joita ensimmäisen kertaluvun predikaattilogiikka ei pysty ilmaisemaan. Itse asiassa on osoitettu, että ensimmäisen kertaluvun predikaattilogiikka kykenee ilmaisemaan vain verkkojen lokaaleja ominaisuuksia \cite{Han65}. Hanf käytti tässä todistuksessaan EF-peliä, tarkemmin Fraïssén algebrallista versiota siitä. Sama lokaalisuus on osoitettu myöhemmin myös toisella metodilla, kvanttorien eliminoinnilla \cite{Gai82}. Nämä tulokset ovat motivoineet pyrkimyksiä kehittää ensimmäisen kertaluvun predikaattilogiikan laajennuksia verkoille, samoin kuin kehittämään näille omia EF-pelejä ilmaisuvoiman tutkimiseen. Tällaisia tutkimuksia on tehty muun muassa monadiselle toisen kertaluvun predikaattilogiikalle \cite{Fag75} \cite{Fag93}, transitiivisen sulkeuman logiikalle \cite{Gra92} ja erilaisille kiintopistelogiikoille \cite{Bos93}.

Vaikka ensimmäisen kertaluvun predikaattilogiikka on hyvin rajoittunut kieli esimerkiksi verkkojen ominaisuuksien ilmaisemiseen, niin kuitenkin sillä voi joitain hyödyllisiäkin kyselyitä verkkojen suhteen ilmaista. Esimerkiksi seuraavat lauseet ovat ilmaistavissa ensimmäisen kertaluvun predikaattilogiikalla (oletetaan, että $E$ on relaatio joka ilmaisee verkon solmujen välistä kaarta ja symbolit $x, y, z_i, q$ ovat solmuja):

\begin{itemize}
\item ``solmulla $x$ on vähintään kaksi toisistaan eroavaa naapuria'' \[(\exists y)(\exists q)(\neg(y = q) \land E(x, y) \land E(x, q))\]
\item ``jokaisella solmulla $x$ on vähintään kaksi toisistaan eroavaa naapuria'' \[(\forall x)(\exists y)(\exists q)(\neg(y = q)\land E(x, y) \land E(x, q))\]
\item ``on olemassa polku solmusta $x$ solmuun $y$ jonka pituus on $3$'' \[(\exists z_1)(\exists z_2)(E(x, z_1)\land E(z_1, z_2) \land E(z_2, y))\]
\end{itemize}

Verkkojen ohella EF-pelit ovat olleet hyödyllisiä ensimmäisen kertaluvun predikaattilogiikan ja formaalien kielten teorian määriteltävyyskysymysten välisen suhteen tutkimisessa. Erityisesti \textit{tähtivapaat säännölliset kielet} (star-free regular languages) ovat olleet mielenkiinnon kohteena.

\subsection{Kongruenssiapulauseet}
Kieltä kutsutaan \textit{tähtivapaaaksi}, jos sen pystyy kuvailemaan säännöllisenä lausekkeena, joka on konstruoitu aakkoston symboleista, tyhjän joukon symbolista ja kaikista muista loogisista operaatioista, kuten konkatenaatiosta ja komplementista paitsi Kleenen tähdestä. Hyvin tunnettu tulos formaalien kielten teoriassa on, että kieli on määriteltävissä ensimmäinen kertaluvun predikaattilogiikan keinoin jos ja vain jos se on tähtivapaa \cite{McN71}. Tämän todistuksessa käytetään yleensä induktiota kvanttorisyvyyden suhteen kuten esimerkiksi Ladner tekee \cite{Lad77}.

\begin{maar}[Kvanttoriaste]
Ensimmäisen kertaluvun predikaattilogiikan kaavan $\varphi$ \textit{kvanttoriaste} $qr(\varphi)$ on sisäkkäisten kvanttorien syvyys ja se määritellään seuraavasti:
\begin{itemize}
\item Jos $\varphi$ on atomikaava, niin $qr(\varphi) = 0$.
\item $qr(\neg \varphi) = qr(\varphi)$.
\item $qr(\varphi_1 \land \varphi_2) = qr(\varphi_1 \lor \varphi_2) = max\{qr(\varphi_1), qr(\varphi_2)\}$
\item $qr(\forall x \varphi) = qr(\exists x \varphi) = qr(\varphi) + 1$.
\end{itemize}
\end{maar}

Esitellään miten EF-peliä käytetään tähtivapaan kielen ja kielen ensimmäisen kertaluvun predikaattilogiikan määriteltävyyden välisen loogisen ekvivalenssin todistamiseen. Edellä mainitun ekvivalenssin todistuksen induktioaskeleen kriittinen kohta on seuraava väite:

\begin{lem}[Kongruenssilemma]
Sanan $w$ yli aakkoston $\Sigma = \{s_1, \ldots, s_k\}$ voi esittää mallina $W = (\{1, \ldots, |w|\}, <, R_1, \ldots, R_k)$, jossa $R_i$ on yksipaikkainen relaatio ja $j \in R_i$ jos ja vain jos $j$:s kirjain sanasta $w$ on $s_i$. Olkoon $s, s', t, t'$ sanoja yli aakkoston $\Sigma$ ja olkoon $S, S', T , T'$ näitä kuvailevia malleja. Tällöin: \[ S \cong_p S' \land T \cong_p T' \Longrightarrow S \cdot T \cong_p S' \cdot T' \]
\end{lem}
Eli jos mallien $S$ ja $S'$ välillä on osittaisisomorfismi sekä mallien $T$ ja $T'$ välillä on osittaisisomorfismi, niin tällöin mallien $S$ ja $T$ konkatenaation ja mallien $S'$ ja $T'$ konkatenaation välillä on osittaisisomorfismi. Mallien konkatenaation voi käsittää tässä tavalliseksi sanojen konkatenaatioksi, koska mallit ovat vain sanojen esitysmuotoja.

Lemman todistus on EF-peliä käyttämällä hyvin suoraviivainen. Oletuksen nojalla Pelaajalla II on voittostrategia peleissä $EF_m(S, S')$ ja $EF_m(T, T')$, joten Pelaajan II voittostrategia pelille $EF_m(S \cdot T, S' \cdot T')$ on kompositio kummastakin oletuksen voittostrategiasta. Siis osille $S$ ja $S'$ käytetään ensimmäisen pelin voittostrategiaa ja osille $T$ ja $T'$ toisen pelin voittostrategiaa.

Kongruenssilemma sanoo, että mallin ominaisuudet määräytyvät osiensa ominaisuuksien mukaan. Täten mallit voidaan myös rakentaa osista ja näiden ominaisuuksista. Kongruenssilemmat ovat tyypillisiä EF-pelien sovelluksia. Näitä on todistettu monille muille logiikoilla sekä sanoja monimutkaisemmille malleille. Esimerkiksi Shelah esittelee esimerkkejä monadisen logiikan kongruenssilemmoista lineaarijärjestyksille \cite{She75}. Wolfgang Thomas käyttää EF-pelejä todistaakseen kongruenssilemman ensimmäisen kertaluvun predikaattilogiikan muunnokselle, jossa kaavat ovat \textit{prenex-normaalimuodossa}, eli kaavat on muokattu niin, että kvanttorit ovat jonona kaavan alussa määrätynlaisessa järjestyksessä, jonka jälkeen seuraa kvanttoreilla sitomaton osuus \cite{Tho84}.

Predikaattilogiikan ja tähtivapaiden lausekkeiden ekvivalenssi on hyvin tunnettu. Thomas ja Lippert esittelevät EF-pelin muunnoksen, konkatenaatiopelin, jonka avulla he esittelevät ensimmäisen kertaluvun predikaattilogiikan ja \textit{relativoitujen tähtivapaiden säännöllisten lausekkeiden} (relativized star-free expressions) välisiä eroja. Tähtivapaa säännöllinen lauseke on relativoitu, jos siihen on lisätty ylimääräinen vakio ja tämä vakio kiinnitetään johonkin kieleen. Tässä työssään he näyttävät että relativoidut tähtivapaat lausekkeet ovat heikompia kuin vastaavat ensimmäisen kertaluvun predikaattilogiikan lauseet \cite{Lip88}.

\subsection{Bisimulaatio}
Yksi tärkeä malliteoreettisten pelien sovellus automaattien ja tilasiirtymäsysteemien teoriassa on Parkin esittelemä \textit{bisimulaation} käsite \cite{Par81}. Bisimulaatiota voi tarkastella eräänlaisena osittaisisomorfismien ``perheenä'', joka vastaa rajoitettua EF-peliä, jossa osittaisisomorfismia on heikennetty niin että kuvauksen ei tarvitse olla enää injektiivinen. Vaikka klassisen, tässä tutkielmassa määritellyn, EF-pelin ja bisimulaation välillä on hyvin läheinen kytkös, niin ne on kehitetty kuitenkin hyvin pitkälti erillään toisistaan. Bisimulaatiota käytetään esimerkiksi Hennessyn ja Milnerin modaalilogiikassa \cite{Hen80}. 

\textit{Hennessy-Milner modaalilogiikkaa} käytetään tilasiirtymäsysteemien omaisuuksien määrittelyyn. Tilasiirtymäsysteemit ovat hyvin paljon automaatteja muistuttavia struktuureja. Modaalilogiikka on propositiologiikan laajennus jossa propositiologiikan kieleen on lisätty uusia operaattoreita, joita kutsutaan \textit{modaalioperaattoreiksi} tai \textit{modaliteeteiksi}. Modaliteetti tarkoittaa jonkin asian tai tapahtuman laatua eli niiden tapaa olla. Modaalilogiikassa sen voi ajatella määrittävän tavan, jolla väite on tosi. Modaliteetin tarkempi määrittely ei ole tämän tutkielman kontekstissa oleellista. Määritellään seuraavaksi bisimulaatio.

\begin{maar}[Bisimulaatio]
Olkoon $M$ ja $M'$ malleja, jotka koostuvat solmujen (tilojen) joukosta $W$, solmujen välisistä relaatioista (kaarista) $R$ ja funktiosta $V$, joka liittää jokaiseen propositiosymboliin $p_i$ joukon $W$ osajoukon $V(p_i)$. Intuitiivisesti ajatellen joukko $V(p_i)$ on niiden tilojen joukko, jossa $p_i$ on tosi. \textit{Bisimulaatio} mallien $M$ ja $M'$ välillä on epätyhjä relaatio $B \subseteq W \times W'$, jolle pätee kaikilla $(w, w') \in B$ seuraavat ehdot:
\begin{itemize}
\item Tilat $w$ ja $w'$ toteuttavat samat propositiosymbolit.
\item Jos $(w, v) \in R$, niin on olemassa $v' \in W'$, jolle pätee $(w', v') \in R'$ ja $(v, v') \in B$.
\item Jos $(w', v') \in R'$, niin on olemassa $v \in W$, jolle pätee $(w, v) \in R$ ja $(v, v') \in B$.
\end{itemize}
Jos on olemassa jokin bisimulaatio $B$ mallien $M$ ja $M'$ välillä ja $(w, w') \in B$ niin sanotaan, että tilat ovat bisimilaarisia.
\end{maar}

Kaksi tilaa ovat siis bisimilaarisia, jos ja vain jos niiden toteuttamat propositiosymbolit sekä tilojen mahdolliset tilasiirtymät vastaavat toisiaan. Yleensä tilat nimetään siten, että niistä ilmenee tilan nimi, sekä tilan mahdollisesti toteuttamat propositiosymbolit. Seuraava kuva havainnollistaa tilojen bisimilaarisuutta:

\begin{center}
\begin{tikzpicture}[-latex ,auto ,node distance =2 cm and 2cm ,on grid ,
semithick ,
state/.style ={ circle ,top color =white , bottom color = processblue!20 ,
draw,processblue , text=blue , minimum width =1 cm}]
\node[state] (x_1) {$w_1 | p_0$};
\node[state] (x_2) [right =of x_1] {$v_1 | p_1$};
\node[state] (x_3) [right =of x_2] {$w_2 | p_0$};
\node[state] (x_4) [right =of x_3] {$v_2 | p_1$};
\node(M) [left =of x_1]%
{$M$};
\node(M') [below =of M]%
{$M'$};
\node[state] (x_5) [below =of x_2] {$w | p_0$};
\node[state] (x_6) [below =of x_3] {$v | p_1$};
\path (x_1) edge node [right] {} (x_2);
\path (x_2) edge node [right] {} (x_3);
\path (x_3) edge node [right] {} (x_4);
\path (x_5) edge [bend left] node [right] {} (x_6);
\path (x_6) edge [bend left] node [left] {} (x_5);
\draw[gray,thin,dashed] (x_1) -- (x_5);
\draw[gray,thin,dashed] (x_3) -- (x_5);
\draw[gray,thin,dashed] (x_2) -- (x_6);
\draw[gray,thin,dashed] (x_4) -- (x_6);
\node[align=center, below] at (2.0,-3.0)%
{Bisimilaariset tilat on yhdistetty katkonuolilla};
\end{tikzpicture}
\end{center}

Bisimulaation käsite mahdollistaa sen tutkimisen että mitä mallien ominaisuuksia on mahdollista kuvailla modaalilogiikan avulla ja mitä taas ei ole mahdollista kuvailla. Bisimulaatio on yksi modaalilogiikan tärkeimpiä työkaluja ja sopii mainiosta ohjelmien verifiointiin \cite{Bla01}. Se onkin yksi verifioinnin ja samanaikaisuusteorian kulmakivistä. Sille löytyy myös käyttöä tekoälytutkimuksessa, lingvistiikassa ja filosofiassa \cite{Bla06}.

\section{Yhteenveto}
Tässä tutkielmassa on tarkasteltu Ehrenfeucht--Fraïssé-pelejä ja näiden sovelluksia eri näkökulmista. Teoreettiselta kannalta tutkielmassa esiteltiin EF-peli, sen säännöt ja esiteltiin esimerkin avulla miten EF-peliä käytännössä pelataan. Lisäksi todistettiin, että Pelaajan II voittostrategia on ekvivalenssirelaatio joten voittostrategian avulla voidaan jakaa malleja ekvivalenssiluokkiin ja täten tarvittaessa samaistaa nämä mallit yhdeksi malliksi. Käytännön kannalta tutkielmassa on esimerkkien avulla esitelty miten EF-peliä sovelletaan määriteltävyyskysymyksissä ja todistuksissa. Kirjallisuutta ja tutkimuksia on esitelty syvällistä aiheeseen tutustumista helpottamaan.

Tutkielmassa esiteltiin myös laajalti EF-peleihin liittyvää matemaattista peruskäsitteistöä, kuten kielet ja mallit sekä isomorfismi ja totuus predikaattilogiikassa. Lisäksi aivan aluksi luotiin pikainen katsaus EF-pelien historiaan ja syihin miksi EF-pelit ovat äärellisten mallien teoriassa niin keskeisessä osassa.



% Write some science here.

% Esimerkkilause ja lähdeviite~\cite{esimerkki}.
 
\newpage
% --- References ---
%
% bibtex is used to generate the bibliography. The babplain style
% will generate numeric references (e.g. [1]) appropriate for theoretical
% computer science. If you need alphanumeric references (e.g [Tur90]), use
%
%\bibliographystyle{babalpha-lf}
%
% instead.

\bibliographystyle{babplain-lf}
\bibliography{references-fi}


% --- Appendices ---

% uncomment the following

% \newpage
% \appendix
% 
% \section{Esimerkkiliite}

\end{document}
