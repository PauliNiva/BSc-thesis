% --- Template for thesis / report with tktltiki2 class ---
% 
% last updated 2013/02/15 for tkltiki2 v1.02

\documentclass[finnish]{tktltiki2}

% tktltiki2 automatically loads babel, so you can simply
% give the language parameter (e.g. finnish, swedish, english, british) as
% a parameter for the class: \documentclass[finnish]{tktltiki2}.
% The information on title and abstract is generated automatically depending on
% the language, see below if you need to change any of these manually.
% 
% Class options:
% - grading                 -- Print labels for grading information on the front page.
% - disablelastpagecounter  -- Disables the automatic generation of page number information
%                              in the abstract. See also \numberofpagesinformation{} command below.
%
% The class also respects the following options of article class:
%   10pt, 11pt, 12pt, final, draft, oneside, twoside,
%   openright, openany, onecolumn, twocolumn, leqno, fleqn
%
% The default font size is 11pt. The paper size used is A4, other sizes are not supported.
%
% rubber: module pdftex

% --- General packages ---

\usepackage[utf8]{inputenc}
\usepackage[T1]{fontenc}
\usepackage{lmodern}
\usepackage{microtype}
\usepackage{amsfonts,amsmath,amssymb,amsthm,booktabs,color,enumitem,graphicx}
\usepackage[pdftex,hidelinks]{hyperref}

% Automatically set the PDF metadata fields
\makeatletter
\AtBeginDocument{\hypersetup{pdftitle = {\@title}, pdfauthor = {\@author}}}
\makeatother

% --- Language-related settings ---
%
% these should be modified according to your language

% babelbib for non-english bibliography using bibtex
\usepackage[fixlanguage]{babelbib}
\selectbiblanguage{finnish}

% add bibliography to the table of contents
\usepackage[nottoc]{tocbibind}
% tocbibind renames the bibliography, use the following to change it back
\settocbibname{Lähteet}

% --- Theorem environment definitions ---

\newtheorem{lau}{Lause}
\newtheorem{lem}[lau]{Lemma}
\newtheorem{kor}[lau]{Korollaari}

\theoremstyle{definition}
\newtheorem{maar}[lau]{Määritelmä}
\newtheorem{ong}{Ongelma}
\newtheorem{alg}[lau]{Algoritmi}
\newtheorem{esim}[lau]{Esimerkki}

\theoremstyle{remark}
\newtheorem*{huom}{Huomautus}


% --- tktltiki2 options ---
%
% The following commands define the information used to generate title and
% abstract pages. The following entries should be always specified:

\title{Predikaattilogiikka ja mallit}
\author{Pauli Niva}
\date{\today}
\level{Kypsyysnäyte}
\abstract{Tiivistelmä.}

% The following can be used to specify keywords and classification of the paper:

\keywords{avainsana 1, avainsana 2, avainsana 3}

% classification according to ACM Computing Classification System (http://www.acm.org/about/class/)
% This is probably mostly relevant for computer scientists
% uncomment the following; contents of \classification will be printed under the abstract with a title
% "ACM Computing Classification System (CCS):"
% \classification{}

% If the automatic page number counting is not working as desired in your case,
% uncomment the following to manually set the number of pages displayed in the abstract page:
%
% \numberofpagesinformation{16 sivua + 10 sivua liitteissä}
%
% If you are not a computer scientist, you will want to uncomment the following by hand and specify
% your department, faculty and subject by hand:
%
% \faculty{Matemaattis-luonnontieteellinen}
% \department{Tietojenkäsittelytieteen laitos}
% \subject{Tietojenkäsittelytiede}
%
% If you are not from the University of Helsinki, then you will most likely want to set these also:
%
% \university{Helsingin Yliopisto}
% \universitylong{HELSINGIN YLIOPISTO --- HELSINGFORS UNIVERSITET --- UNIVERSITY OF HELSINKI} % displayed on the top of the abstract page
% \city{Helsinki}
%


\begin{document}

% --- Front matter ---

\frontmatter      % roman page numbering for front matter

\maketitle        % title page
%\makeabstract     % abstract page

%\tableofcontents  % table of contents

% --- Main matter ---

\mainmatter       % clear page, start arabic page numbering

%\section{Esimerkkiluku}

% Write some science here.
\noindent
``Malli? Mikä ihmeen malli?'' Tämän kysymyksen kuulee hyvin usein tietojenkäsittelytieteilijän suusta. Syynä tähän on todennäköisesti se, että tietojenkäsittelytieteen opetuksessa teoreettisia asioita painotetaan aivan liian vähän.

Mitä nämä \textit{mallit} sitten ovat? Malleja ovat esimerkiksi luonnolliset luvut, reaaliluvut, kompleksiluvut, relaatiotietokannat. Samoin erilaiset tietorakenteet, verkot, erilaiset algebralliset struktuurit, kuten ryhmät, renkaat ja kunnat ovat malleja. Malliksi voidaan kutsua myös kaikkien tätä kypsyysnäytettä lukevien ihmisten joukkoa kaveruussuhteineen -- mitäpä olisi logiikkaa käsittelevä kirjoitus ilman itseensä viittaamista. Mallin käsite on siis kaiken kaikkiaan hyvin geneerinen ja abstrakti.

Otetaan esimerkiksi \textit{järjestetyt kunnat}, tarkemmin sanottuna reaalilukujen ja rationaalilukujen muodostamat systeemit. Kummassakin systeemissä on yhteenlasku, kertolasku, nolla, ykkönen ja järjestys. Kun näitä vastaavat symbolit kerätään joukoksi, saadaan aikaan näiden yhteinen aakkosto. \textit{Aakkosto} on siis mallin skemaattinen esitys, joka kertoo mallin rakenteen pääpiirteet. Yleisesti ottaen mallin aakkosto voi sisältää seuraavanlaisia olioita: a) relaatioita, joita edellä mainitussa aakkostossa edustaa järjestys sekä b) vakioita, joita määrittelemässämme aakkostossa vastaavat ykkönen ja nolla. Lisäksi malli voi sisältää myös c) kuvauksia. Näitä vastaa esimerkkiaakkoston yhteen- ja kertolasku.

Predikaattilogiikka puolestaan on symbolisen logiikan osa-alue, joka tutkii luonnollisen kielen subjekti-predikaatti-muotoisia lauseita mallintavia formaaleja kieliä. Mallit tarjoavat predikaattilogiikalle semantiikan eli merkityksen ja kontekstin, jossa formaalin kieleen lauseet voivat olla tosia tai epätosia. Totuus predikaattilogiikassa tarkoittaa siis totuutta jossakin mallissa.

Malleja on äärettömiä, kuten esimerkiksi reaalilukujen joukko, sekä äärellisiä, kuten esimerkiksi jokin määrätty relaatiotietokanta. Äärettömiä malleja on tutkittu hyvin pitkään, mutta tietotekniikan nousu on luonut tarpeen rajoittua äärettömistä malleista äärellisiin malleihin. Äärellisten mallien teorian kehittelyn ohessa nopeasti huomattiin, että aikojen saatossa äärettömille malleille kehitetyt työkalut eivät enää pääsääntöisesti toimineetkaan äärellisillä malleilla. Yksi harvoista työkaluista, joka kuitenkin toimii myös äärellisten mallien tapauksessa, on Ehrenfeucht--Fraïssé-peli.

Monet laskennan vaativuusongelmat sekä tietokantateorian ongelmat voidaan uudelleen muotoilla matemaattisen logiikan ongelmina, kunhan rajoitutaan äärellisiin malleihin. Malleja voi ajatella esimerkiksi tietokoneohjelmien syötteinä eli relationaalisina tietokantoina. Kun tätä ideaa kehitellään eteenpäin, päädytään deskriptiiviseen vaativuusteoriaan. Yleisemmällä tasolla voidaan sanoa, että jos ongelma voidaan muotoilla predikaattilogiikan lauseeksi, voidaan ongelman ratkaisuun käyttää predikaattilogiikan työkaluja, kuten Ehrenfeucht--Fraïssé-peliä.

Ehrenfeucht--Fraïssé-peli tarjoaa keinon kahden mallin samankaltaisuuden mittaamiseen. Muita samankaltaisuuden mittareita on muun muassa isomorfismi, mutta mallien samankaltaisuuden mittaamiseen se on yleensä aivan liian karkea työkalu. Keskeinen ero Ehrenfeucht--Fraïssé-pelissä isomorfismiin verrattuna on se, että edellisen tapauksessa mallien tutkimiseen riittää rajoitetun pistemäärän tarkastelu. Mallien tarkasteluun vaadittava aika on siis rajattu, jos ajatellaan että yksi Ehrenfeucht--Fraïssé-pelin kierros kuluttaa yhden aikayksikön.

Ehrenfeucht--Fraïssé-peli voidaan muotoilla algebralliseen muotoon tai peliteoreettiseen muotoon. Fraïssé kehitti vuonna 1954 Fraïssén systeemin mallien samankaltaisuuden mittaamista varten. Tämä on edellä mainittu Ehrenfeucht--Fraïssé-pelin algebrallinen muotoilu. Ehrenfeucht puki tämän systeemin peliteoreettiseen muotoon 1960-luvun alussa. Vaikka peli on varsinaisesti peräisin Ehrenfeuchtilta, muotoilujen yhtäpitävyyden vuoksi on tapana puhua Ehrenfeucht--Fraïssé-pelistä.

Predikaattilogiikassa on muitakin työkaluja mallien samankaltaisuuden tarkasteluun kuin Ehrenfeucht--Fraïssé-peli. Nämä ovat kuitenkin suurimmaksi osaksi vain muunnoksia Ehrenfeucht--Fraïssé-pelistä. Tällaisia pelejä ovat muun muassa Barwisen peli ja helmipeli. Esimerkiksi helmipeli on Ehfenfeucht-Fraïsse-pelin muunnos, jossa aikaresurssin sijaan rajoitetaan muistiresurssia. Tämä toteutetaan siten, että toisella pelaajista on oikeus peruuttaa siirtonsa.

Koska mallit tarjoavat keinon tarkastella predikaattilogiikan lauseiden totuutta ja Ehrenfeucht--Fraïssé-pelillä voidaan mitata mallien samankaltaisuutta, soveltuvat nämä yhdessä erinomaisesti  predikaattilogiikan ilmaisuvoiman mittaamiseen. Tämä taas vuorostaan mahdollistaa monien teoreettisen tietojenkäsittelytieteen ongelmien ratkaisun, mutta nämä ovatkin jo uusia aiheita itsessään.
% --- References ---
%
% bibtex is used to generate the bibliography. The babplain style
% will generate numeric references (e.g. [1]) appropriate for theoretical
% computer science. If you need alphanumeric references (e.g [Tur90]), use
%
% \bibliographystyle{babalpha-lf}
%
% instead.

%\bibliographystyle{babplain-lf}
%\bibliography{references-fi}


% --- Appendices ---

% uncomment the following

% \newpage
% \appendix
% 
% \section{Esimerkkiliite}

\end{document}
